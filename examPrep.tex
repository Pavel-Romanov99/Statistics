% Options for packages loaded elsewhere
\PassOptionsToPackage{unicode}{hyperref}
\PassOptionsToPackage{hyphens}{url}
%
\documentclass[
]{article}
\usepackage{amsmath,amssymb}
\usepackage{lmodern}
\usepackage{ifxetex,ifluatex}
\ifnum 0\ifxetex 1\fi\ifluatex 1\fi=0 % if pdftex
  \usepackage[T1]{fontenc}
  \usepackage[utf8]{inputenc}
  \usepackage{textcomp} % provide euro and other symbols
\else % if luatex or xetex
  \usepackage{unicode-math}
  \defaultfontfeatures{Scale=MatchLowercase}
  \defaultfontfeatures[\rmfamily]{Ligatures=TeX,Scale=1}
\fi
% Use upquote if available, for straight quotes in verbatim environments
\IfFileExists{upquote.sty}{\usepackage{upquote}}{}
\IfFileExists{microtype.sty}{% use microtype if available
  \usepackage[]{microtype}
  \UseMicrotypeSet[protrusion]{basicmath} % disable protrusion for tt fonts
}{}
\makeatletter
\@ifundefined{KOMAClassName}{% if non-KOMA class
  \IfFileExists{parskip.sty}{%
    \usepackage{parskip}
  }{% else
    \setlength{\parindent}{0pt}
    \setlength{\parskip}{6pt plus 2pt minus 1pt}}
}{% if KOMA class
  \KOMAoptions{parskip=half}}
\makeatother
\usepackage{xcolor}
\IfFileExists{xurl.sty}{\usepackage{xurl}}{} % add URL line breaks if available
\IfFileExists{bookmark.sty}{\usepackage{bookmark}}{\usepackage{hyperref}}
\hypersetup{
  hidelinks,
  pdfcreator={LaTeX via pandoc}}
\urlstyle{same} % disable monospaced font for URLs
\usepackage[margin=1in]{geometry}
\usepackage{color}
\usepackage{fancyvrb}
\newcommand{\VerbBar}{|}
\newcommand{\VERB}{\Verb[commandchars=\\\{\}]}
\DefineVerbatimEnvironment{Highlighting}{Verbatim}{commandchars=\\\{\}}
% Add ',fontsize=\small' for more characters per line
\usepackage{framed}
\definecolor{shadecolor}{RGB}{248,248,248}
\newenvironment{Shaded}{\begin{snugshade}}{\end{snugshade}}
\newcommand{\AlertTok}[1]{\textcolor[rgb]{0.94,0.16,0.16}{#1}}
\newcommand{\AnnotationTok}[1]{\textcolor[rgb]{0.56,0.35,0.01}{\textbf{\textit{#1}}}}
\newcommand{\AttributeTok}[1]{\textcolor[rgb]{0.77,0.63,0.00}{#1}}
\newcommand{\BaseNTok}[1]{\textcolor[rgb]{0.00,0.00,0.81}{#1}}
\newcommand{\BuiltInTok}[1]{#1}
\newcommand{\CharTok}[1]{\textcolor[rgb]{0.31,0.60,0.02}{#1}}
\newcommand{\CommentTok}[1]{\textcolor[rgb]{0.56,0.35,0.01}{\textit{#1}}}
\newcommand{\CommentVarTok}[1]{\textcolor[rgb]{0.56,0.35,0.01}{\textbf{\textit{#1}}}}
\newcommand{\ConstantTok}[1]{\textcolor[rgb]{0.00,0.00,0.00}{#1}}
\newcommand{\ControlFlowTok}[1]{\textcolor[rgb]{0.13,0.29,0.53}{\textbf{#1}}}
\newcommand{\DataTypeTok}[1]{\textcolor[rgb]{0.13,0.29,0.53}{#1}}
\newcommand{\DecValTok}[1]{\textcolor[rgb]{0.00,0.00,0.81}{#1}}
\newcommand{\DocumentationTok}[1]{\textcolor[rgb]{0.56,0.35,0.01}{\textbf{\textit{#1}}}}
\newcommand{\ErrorTok}[1]{\textcolor[rgb]{0.64,0.00,0.00}{\textbf{#1}}}
\newcommand{\ExtensionTok}[1]{#1}
\newcommand{\FloatTok}[1]{\textcolor[rgb]{0.00,0.00,0.81}{#1}}
\newcommand{\FunctionTok}[1]{\textcolor[rgb]{0.00,0.00,0.00}{#1}}
\newcommand{\ImportTok}[1]{#1}
\newcommand{\InformationTok}[1]{\textcolor[rgb]{0.56,0.35,0.01}{\textbf{\textit{#1}}}}
\newcommand{\KeywordTok}[1]{\textcolor[rgb]{0.13,0.29,0.53}{\textbf{#1}}}
\newcommand{\NormalTok}[1]{#1}
\newcommand{\OperatorTok}[1]{\textcolor[rgb]{0.81,0.36,0.00}{\textbf{#1}}}
\newcommand{\OtherTok}[1]{\textcolor[rgb]{0.56,0.35,0.01}{#1}}
\newcommand{\PreprocessorTok}[1]{\textcolor[rgb]{0.56,0.35,0.01}{\textit{#1}}}
\newcommand{\RegionMarkerTok}[1]{#1}
\newcommand{\SpecialCharTok}[1]{\textcolor[rgb]{0.00,0.00,0.00}{#1}}
\newcommand{\SpecialStringTok}[1]{\textcolor[rgb]{0.31,0.60,0.02}{#1}}
\newcommand{\StringTok}[1]{\textcolor[rgb]{0.31,0.60,0.02}{#1}}
\newcommand{\VariableTok}[1]{\textcolor[rgb]{0.00,0.00,0.00}{#1}}
\newcommand{\VerbatimStringTok}[1]{\textcolor[rgb]{0.31,0.60,0.02}{#1}}
\newcommand{\WarningTok}[1]{\textcolor[rgb]{0.56,0.35,0.01}{\textbf{\textit{#1}}}}
\usepackage{graphicx}
\makeatletter
\def\maxwidth{\ifdim\Gin@nat@width>\linewidth\linewidth\else\Gin@nat@width\fi}
\def\maxheight{\ifdim\Gin@nat@height>\textheight\textheight\else\Gin@nat@height\fi}
\makeatother
% Scale images if necessary, so that they will not overflow the page
% margins by default, and it is still possible to overwrite the defaults
% using explicit options in \includegraphics[width, height, ...]{}
\setkeys{Gin}{width=\maxwidth,height=\maxheight,keepaspectratio}
% Set default figure placement to htbp
\makeatletter
\def\fps@figure{htbp}
\makeatother
\setlength{\emergencystretch}{3em} % prevent overfull lines
\providecommand{\tightlist}{%
  \setlength{\itemsep}{0pt}\setlength{\parskip}{0pt}}
\setcounter{secnumdepth}{-\maxdimen} % remove section numbering
\ifluatex
  \usepackage{selnolig}  % disable illegal ligatures
\fi

\author{}
\date{\vspace{-2.5em}}

\begin{document}

`--- title: ``ExamPrep'' output: pdf\_document ---

От упражнение 1

\begin{Shaded}
\begin{Highlighting}[]
\CommentTok{\#1.}
\CommentTok{\#create a vector}
\NormalTok{vect }\OtherTok{\textless{}{-}} \FunctionTok{c}\NormalTok{(}\DecValTok{8}\NormalTok{,}\DecValTok{3}\NormalTok{,}\DecValTok{8}\NormalTok{,}\DecValTok{7}\NormalTok{,}\DecValTok{15}\NormalTok{,}\DecValTok{9}\NormalTok{,}\DecValTok{12}\NormalTok{,}\DecValTok{4}\NormalTok{,}\DecValTok{9}\NormalTok{,}\DecValTok{10}\NormalTok{,}\DecValTok{5}\NormalTok{,}\DecValTok{1}\NormalTok{)}
\CommentTok{\#create a 4x3 matrix}
\NormalTok{m }\OtherTok{\textless{}{-}} \FunctionTok{matrix}\NormalTok{(vect, }\AttributeTok{nrow =} \DecValTok{4}\NormalTok{, }\AttributeTok{ncol =} \DecValTok{3}\NormalTok{)}
\CommentTok{\#adding a column}
\NormalTok{m1 }\OtherTok{\textless{}{-}} \FunctionTok{cbind}\NormalTok{(m, }\FunctionTok{c}\NormalTok{(}\DecValTok{1}\NormalTok{,}\DecValTok{3}\NormalTok{,}\DecValTok{5}\NormalTok{,}\DecValTok{7}\NormalTok{))}

\CommentTok{\#indexes of the first column}
\NormalTok{ordered }\OtherTok{\textless{}{-}} \FunctionTok{order}\NormalTok{(m1[,}\DecValTok{1}\NormalTok{], }\AttributeTok{decreasing =} \ConstantTok{FALSE}\NormalTok{)}
\CommentTok{\#ordered matrix}
\NormalTok{ordered\_by\_first\_column }\OtherTok{\textless{}{-}}\NormalTok{ m1[ordered,]}

\CommentTok{\#indexes of the first two columns}
\NormalTok{ordered2 }\OtherTok{\textless{}{-}} \FunctionTok{order}\NormalTok{(m1[,}\DecValTok{1}\NormalTok{], m1[,}\DecValTok{2}\NormalTok{], }\AttributeTok{decreasing =} \ConstantTok{FALSE}\NormalTok{)}
\CommentTok{\#ordered matrix}
\NormalTok{ordered\_by\_two\_columns }\OtherTok{\textless{}{-}}\NormalTok{ m1[ordered2, ]}
\end{Highlighting}
\end{Shaded}

\begin{Shaded}
\begin{Highlighting}[]
\CommentTok{\#2.}
\CommentTok{\#most and least expensive in 2000}
\NormalTok{most }\OtherTok{\textless{}{-}} \FunctionTok{which.max}\NormalTok{(homedata}\SpecialCharTok{$}\NormalTok{y2000)}
\NormalTok{least }\OtherTok{\textless{}{-}} \FunctionTok{which.min}\NormalTok{(homedata}\SpecialCharTok{$}\NormalTok{y2000)}

\CommentTok{\#prices in 1970}
\NormalTok{homedata}\SpecialCharTok{$}\NormalTok{y1970[most]}
\end{Highlighting}
\end{Shaded}

\begin{verbatim}
## [1] 198900
\end{verbatim}

\begin{Shaded}
\begin{Highlighting}[]
\NormalTok{homedata}\SpecialCharTok{$}\NormalTok{y1970[least]}
\end{Highlighting}
\end{Shaded}

\begin{verbatim}
## [1] 10000
\end{verbatim}

\begin{Shaded}
\begin{Highlighting}[]
\CommentTok{\#top five most expensive houses in 2000}
\NormalTok{ordered\_five }\OtherTok{\textless{}{-}}\NormalTok{ homedata}\SpecialCharTok{$}\NormalTok{y2000[}\FunctionTok{order}\NormalTok{(homedata}\SpecialCharTok{$}\NormalTok{y2000, }\AttributeTok{decreasing =}\NormalTok{ T)]}

\NormalTok{top\_five }\OtherTok{\textless{}{-}} \FunctionTok{head}\NormalTok{(ordered\_five ,}\DecValTok{5}\NormalTok{)}

\CommentTok{\#средната цена на 5те най{-}скъпи от 2000, но на техните цени от 1970}
\NormalTok{mean\_top\_five }\OtherTok{\textless{}{-}} \FunctionTok{mean}\NormalTok{(}\FunctionTok{head}\NormalTok{(homedata}\SpecialCharTok{$}\NormalTok{y1970[}\FunctionTok{order}\NormalTok{(homedata}\SpecialCharTok{$}\NormalTok{y2000, }\AttributeTok{decreasing =}\NormalTok{ T)], }\DecValTok{5}\NormalTok{))}

\CommentTok{\#къщите, чийто цена е намаляла през 2000г.}
\NormalTok{lowered }\OtherTok{\textless{}{-}}\NormalTok{ homedata}\SpecialCharTok{$}\NormalTok{y2000[}\FunctionTok{which}\NormalTok{(homedata}\SpecialCharTok{$}\NormalTok{y2000 }\SpecialCharTok{\textless{}}\NormalTok{ homedata}\SpecialCharTok{$}\NormalTok{y1970)]}

\NormalTok{percent\_increase }\OtherTok{\textless{}{-}} \FunctionTok{head}\NormalTok{(}\FunctionTok{order}\NormalTok{(((homedata}\SpecialCharTok{$}\NormalTok{y2000 }\SpecialCharTok{{-}}\NormalTok{ homedata}\SpecialCharTok{$}\NormalTok{y1970) }\SpecialCharTok{/}\NormalTok{ homedata}\SpecialCharTok{$}\NormalTok{y1970), }
                           \AttributeTok{decreasing =}\NormalTok{ T), }\DecValTok{10}\NormalTok{)}

\NormalTok{top\_ten\_increase }\OtherTok{\textless{}{-}}\NormalTok{ homedata}\SpecialCharTok{$}\NormalTok{y2000[percent\_increase]}
\end{Highlighting}
\end{Shaded}

\begin{Shaded}
\begin{Highlighting}[]
\CommentTok{\#3.}
\CommentTok{\#number of men}
\NormalTok{num\_men }\OtherTok{\textless{}{-}} \FunctionTok{nrow}\NormalTok{(survey[survey}\SpecialCharTok{$}\NormalTok{Sex }\SpecialCharTok{==} \StringTok{\textquotesingle{}Male\textquotesingle{}}\NormalTok{, ])}

\CommentTok{\#number of men smokers}
\NormalTok{num\_men\_smokers }\OtherTok{\textless{}{-}} \FunctionTok{nrow}\NormalTok{(survey[survey}\SpecialCharTok{$}\NormalTok{Sex }\SpecialCharTok{==} \StringTok{\textquotesingle{}Male\textquotesingle{}} \SpecialCharTok{\&}\NormalTok{ survey}\SpecialCharTok{$}\NormalTok{Smoke }\SpecialCharTok{!=} \StringTok{\textquotesingle{}Never\textquotesingle{}}\NormalTok{, ])}

\CommentTok{\#mean height of all men}
\FunctionTok{mean}\NormalTok{(survey}\SpecialCharTok{$}\NormalTok{Height[survey}\SpecialCharTok{$}\NormalTok{Sex }\SpecialCharTok{==} \StringTok{\textquotesingle{}Male\textquotesingle{}}\NormalTok{], }\AttributeTok{na.rm =}\NormalTok{ T)}
\end{Highlighting}
\end{Shaded}

\begin{verbatim}
## [1] 178.826
\end{verbatim}

\begin{Shaded}
\begin{Highlighting}[]
\CommentTok{\#height and sex of the top 6 youngest students}
\NormalTok{youngest }\OtherTok{\textless{}{-}} \FunctionTok{head}\NormalTok{(}\FunctionTok{order}\NormalTok{(survey}\SpecialCharTok{$}\NormalTok{Age), }\DecValTok{6}\NormalTok{)}

\NormalTok{survey}\SpecialCharTok{$}\NormalTok{Sex[youngest]}
\end{Highlighting}
\end{Shaded}

\begin{verbatim}
## [1] Male   Male   Female Female Female Female
## Levels: Female Male
\end{verbatim}

\begin{Shaded}
\begin{Highlighting}[]
\NormalTok{survey}\SpecialCharTok{$}\NormalTok{Height[youngest]}
\end{Highlighting}
\end{Shaded}

\begin{verbatim}
## [1]     NA     NA     NA 160.00 172.00 170.18
\end{verbatim}

От упражнение 2

\begin{Shaded}
\begin{Highlighting}[]
\CommentTok{\#1.}
\CommentTok{\#случайно избран човек да се окаже пушач}
\NormalTok{survey}\SpecialCharTok{$}\NormalTok{Smoke }\SpecialCharTok{\%\textgreater{}\%} \FunctionTok{table}\NormalTok{() }\SpecialCharTok{\%\textgreater{}\%} \FunctionTok{prop.table}\NormalTok{()}
\end{Highlighting}
\end{Shaded}

\begin{verbatim}
## .
##      Heavy      Never      Occas      Regul 
## 0.04661017 0.80084746 0.08050847 0.07203390
\end{verbatim}

\begin{Shaded}
\begin{Highlighting}[]
\CommentTok{\#случайно избран мъж да се окаже редовно пушещ}
\FunctionTok{table}\NormalTok{(survey}\SpecialCharTok{$}\NormalTok{Smoke, survey}\SpecialCharTok{$}\NormalTok{Sex) }\SpecialCharTok{\%\textgreater{}\%} \FunctionTok{prop.table}\NormalTok{()}
\end{Highlighting}
\end{Shaded}

\begin{verbatim}
##        
##             Female       Male
##   Heavy 0.02127660 0.02553191
##   Never 0.42127660 0.37872340
##   Occas 0.03829787 0.04255319
##   Regul 0.02127660 0.05106383
\end{verbatim}

\begin{Shaded}
\begin{Highlighting}[]
\CommentTok{\#the same as}
\NormalTok{smoking\_men }\OtherTok{\textless{}{-}} \FunctionTok{nrow}\NormalTok{(survey[survey}\SpecialCharTok{$}\NormalTok{Sex }\SpecialCharTok{==} \StringTok{\textquotesingle{}Male\textquotesingle{}} \SpecialCharTok{\&}\NormalTok{ survey}\SpecialCharTok{$}\NormalTok{Smoke }\SpecialCharTok{==} \StringTok{\textquotesingle{}Regul\textquotesingle{}}\NormalTok{, ])}
\NormalTok{smoking\_men }\SpecialCharTok{/} \FunctionTok{nrow}\NormalTok{(survey)}
\end{Highlighting}
\end{Shaded}

\begin{verbatim}
## [1] 0.05485232
\end{verbatim}

\begin{Shaded}
\begin{Highlighting}[]
\CommentTok{\#случаен мъж да се окаже редовен пушач.Стойността на всяка клетка се дели на сумата}
\CommentTok{\#от редовете}
\FunctionTok{prop.table}\NormalTok{(}\FunctionTok{table}\NormalTok{(survey}\SpecialCharTok{$}\NormalTok{Sex, survey}\SpecialCharTok{$}\NormalTok{Smoke), }\DecValTok{1}\NormalTok{)}
\end{Highlighting}
\end{Shaded}

\begin{verbatim}
##         
##               Heavy      Never      Occas      Regul
##   Female 0.04237288 0.83898305 0.07627119 0.04237288
##   Male   0.05128205 0.76068376 0.08547009 0.10256410
\end{verbatim}

\begin{Shaded}
\begin{Highlighting}[]
\CommentTok{\#случаен редовен пушач да се окаже мъж. Клетката се дели на сумата от колоните}
\FunctionTok{prop.table}\NormalTok{(}\FunctionTok{table}\NormalTok{(survey}\SpecialCharTok{$}\NormalTok{Sex, survey}\SpecialCharTok{$}\NormalTok{Smoke), }\DecValTok{2}\NormalTok{)}
\end{Highlighting}
\end{Shaded}

\begin{verbatim}
##         
##              Heavy     Never     Occas     Regul
##   Female 0.4545455 0.5265957 0.4736842 0.2941176
##   Male   0.5454545 0.4734043 0.5263158 0.7058824
\end{verbatim}

\begin{Shaded}
\begin{Highlighting}[]
\CommentTok{\#2.}
\CommentTok{\#направете графики за пушачите и за пола}

\CommentTok{\#графики за пушенето}
\FunctionTok{pie}\NormalTok{(}\FunctionTok{table}\NormalTok{(survey}\SpecialCharTok{$}\NormalTok{Smoke))}
\end{Highlighting}
\end{Shaded}

\includegraphics{examPrep_files/figure-latex/unnamed-chunk-5-1.pdf}

\begin{Shaded}
\begin{Highlighting}[]
\FunctionTok{barplot}\NormalTok{(}\FunctionTok{table}\NormalTok{(survey}\SpecialCharTok{$}\NormalTok{Smoke))}
\end{Highlighting}
\end{Shaded}

\includegraphics{examPrep_files/figure-latex/unnamed-chunk-5-2.pdf}

\begin{Shaded}
\begin{Highlighting}[]
\CommentTok{\#графика за пушенето и пола}
\FunctionTok{barplot}\NormalTok{(}\FunctionTok{table}\NormalTok{(survey}\SpecialCharTok{$}\NormalTok{Smoke, survey}\SpecialCharTok{$}\NormalTok{Sex), }\AttributeTok{beside =}\NormalTok{ T, }\AttributeTok{legend =}\NormalTok{ T)}
\end{Highlighting}
\end{Shaded}

\includegraphics{examPrep_files/figure-latex/unnamed-chunk-5-3.pdf}

\begin{Shaded}
\begin{Highlighting}[]
\CommentTok{\#3.}
\CommentTok{\#за да разделим някаква информация на интервали, които ние искаме ползваме cut}
\NormalTok{groups }\OtherTok{\textless{}{-}} \FunctionTok{cut}\NormalTok{(survey}\SpecialCharTok{$}\NormalTok{Age, }\FunctionTok{c}\NormalTok{(}\DecValTok{0}\NormalTok{, }\DecValTok{20}\NormalTok{, }\DecValTok{25}\NormalTok{, }\DecValTok{100}\NormalTok{), }\FunctionTok{c}\NormalTok{(}\StringTok{\textquotesingle{}teenagers\textquotesingle{}}\NormalTok{, }\StringTok{\textquotesingle{}students\textquotesingle{}}\NormalTok{, }\StringTok{\textquotesingle{}adults\textquotesingle{}}\NormalTok{))}

\CommentTok{\#правим го на графика}
\FunctionTok{table}\NormalTok{(groups, survey}\SpecialCharTok{$}\NormalTok{Smoke) }\SpecialCharTok{\%\textgreater{}\%}
  \FunctionTok{barplot}\NormalTok{(}\AttributeTok{legend =}\NormalTok{ T, }\AttributeTok{beside =}\NormalTok{ T)}
\end{Highlighting}
\end{Shaded}

\includegraphics{examPrep_files/figure-latex/unnamed-chunk-6-1.pdf}

\begin{Shaded}
\begin{Highlighting}[]
\CommentTok{\#4.}
\NormalTok{s }\OtherTok{\textless{}{-}} \FunctionTok{sd}\NormalTok{(survey}\SpecialCharTok{$}\NormalTok{Height, }\AttributeTok{na.rm =}\NormalTok{ T)}

\NormalTok{med }\OtherTok{\textless{}{-}} \FunctionTok{median}\NormalTok{(survey}\SpecialCharTok{$}\NormalTok{Height,}\AttributeTok{na.rm =}\NormalTok{ T)}

\NormalTok{m }\OtherTok{\textless{}{-}} \FunctionTok{mean}\NormalTok{(survey}\SpecialCharTok{$}\NormalTok{Height, }\AttributeTok{na.rm =}\NormalTok{ T)}

\FunctionTok{quantile}\NormalTok{(survey}\SpecialCharTok{$}\NormalTok{Height, }\AttributeTok{na.rm =}\NormalTok{ T)}
\end{Highlighting}
\end{Shaded}

\begin{verbatim}
##   0%  25%  50%  75% 100% 
##  150  165  171  180  200
\end{verbatim}

\begin{Shaded}
\begin{Highlighting}[]
\CommentTok{\#брой различаващи се от средната височина с неповече от 2 стандартни отклонения}
\FunctionTok{cut}\NormalTok{(survey}\SpecialCharTok{$}\NormalTok{Height, }\FunctionTok{c}\NormalTok{(}\DecValTok{0}\NormalTok{, m }\SpecialCharTok{{-}}\NormalTok{ s, m }\SpecialCharTok{+}\NormalTok{ s, }\DecValTok{300}\NormalTok{)) }\SpecialCharTok{\%\textgreater{}\%}
  \FunctionTok{table}\NormalTok{()}
\end{Highlighting}
\end{Shaded}

\begin{verbatim}
## .
##   (0,163] (163,182] (182,300] 
##        28       143        38
\end{verbatim}

От упражение 3

\begin{Shaded}
\begin{Highlighting}[]
\CommentTok{\#1.}
\CommentTok{\#графика според височината и пола}
\FunctionTok{boxplot}\NormalTok{(survey}\SpecialCharTok{$}\NormalTok{Height }\SpecialCharTok{\textasciitilde{}}\NormalTok{ survey}\SpecialCharTok{$}\NormalTok{Sex)}
\end{Highlighting}
\end{Shaded}

\includegraphics{examPrep_files/figure-latex/unnamed-chunk-8-1.pdf}

\begin{Shaded}
\begin{Highlighting}[]
\CommentTok{\#хистограма според височината и имаме плътността}
\FunctionTok{hist}\NormalTok{(survey}\SpecialCharTok{$}\NormalTok{Height, }\AttributeTok{probability =}\NormalTok{ T)}
  \FunctionTok{lines}\NormalTok{(}\FunctionTok{density}\NormalTok{(survey}\SpecialCharTok{$}\NormalTok{Height, }\AttributeTok{na.rm =}\NormalTok{ T))}
\end{Highlighting}
\end{Shaded}

\includegraphics{examPrep_files/figure-latex/unnamed-chunk-8-2.pdf}

\begin{Shaded}
\begin{Highlighting}[]
\CommentTok{\#хистограма според височината на мъжете и имаме линия за плътност}
\NormalTok{men }\OtherTok{\textless{}{-}}\NormalTok{ survey}\SpecialCharTok{$}\NormalTok{Height[survey}\SpecialCharTok{$}\NormalTok{Sex }\SpecialCharTok{==} \StringTok{\textquotesingle{}Male\textquotesingle{}}\NormalTok{]}
\NormalTok{women }\OtherTok{\textless{}{-}}\NormalTok{ survey}\SpecialCharTok{$}\NormalTok{Height[survey}\SpecialCharTok{$}\NormalTok{Sex }\SpecialCharTok{==} \StringTok{\textquotesingle{}Female\textquotesingle{}}\NormalTok{]}

\FunctionTok{hist}\NormalTok{(men, }\AttributeTok{probability =}\NormalTok{ T)}
  \FunctionTok{lines}\NormalTok{(}\FunctionTok{density}\NormalTok{(men, }\AttributeTok{na.rm =}\NormalTok{ T))}
\end{Highlighting}
\end{Shaded}

\includegraphics{examPrep_files/figure-latex/unnamed-chunk-8-3.pdf}

\begin{Shaded}
\begin{Highlighting}[]
\CommentTok{\#графика за плътностите на височините на двата пола}
\FunctionTok{plot}\NormalTok{(}\FunctionTok{density}\NormalTok{(men, }\AttributeTok{na.rm =}\NormalTok{ T), }\AttributeTok{col=}\StringTok{\textquotesingle{}red\textquotesingle{}}\NormalTok{)}
\FunctionTok{lines}\NormalTok{(}\FunctionTok{density}\NormalTok{(women, }\AttributeTok{na.rm =}\NormalTok{ T), }\AttributeTok{col=}\StringTok{\textquotesingle{}blue\textquotesingle{}}\NormalTok{)}
\end{Highlighting}
\end{Shaded}

\includegraphics{examPrep_files/figure-latex/unnamed-chunk-8-4.pdf}

\begin{Shaded}
\begin{Highlighting}[]
\CommentTok{\#2.Histogram for the pulse of the students including the density}
\FunctionTok{hist}\NormalTok{(survey}\SpecialCharTok{$}\NormalTok{Pulse, }\AttributeTok{probability =}\NormalTok{ T)}
\FunctionTok{lines}\NormalTok{(}\FunctionTok{density}\NormalTok{(survey}\SpecialCharTok{$}\NormalTok{Pulse, }\AttributeTok{na.rm =}\NormalTok{ T))}
\end{Highlighting}
\end{Shaded}

\includegraphics{examPrep_files/figure-latex/unnamed-chunk-9-1.pdf}

\begin{Shaded}
\begin{Highlighting}[]
\CommentTok{\#3.}
\CommentTok{\#графиките за къщите от 1970 и 2000г.}
\FunctionTok{hist}\NormalTok{(homedata}\SpecialCharTok{$}\NormalTok{y1970)}
\FunctionTok{lines}\NormalTok{(}\FunctionTok{density}\NormalTok{(homedata}\SpecialCharTok{$}\NormalTok{y1970, }\AttributeTok{na.rm =}\NormalTok{ T))}
\end{Highlighting}
\end{Shaded}

\includegraphics{examPrep_files/figure-latex/unnamed-chunk-10-1.pdf}

\begin{Shaded}
\begin{Highlighting}[]
\FunctionTok{hist}\NormalTok{(homedata}\SpecialCharTok{$}\NormalTok{y2000)}
\FunctionTok{lines}\NormalTok{(}\FunctionTok{density}\NormalTok{(homedata}\SpecialCharTok{$}\NormalTok{y2000, }\AttributeTok{na.rm =}\NormalTok{ T))}
\end{Highlighting}
\end{Shaded}

\includegraphics{examPrep_files/figure-latex/unnamed-chunk-10-2.pdf}

\begin{Shaded}
\begin{Highlighting}[]
\CommentTok{\#сравняваме цените на къщите през 1970 и 2000г. и тяхната корелация}
\FunctionTok{boxplot}\NormalTok{(homedata) }
\end{Highlighting}
\end{Shaded}

\includegraphics{examPrep_files/figure-latex/unnamed-chunk-10-3.pdf}

\begin{Shaded}
\begin{Highlighting}[]
\NormalTok{correlation }\OtherTok{\textless{}{-}} \FunctionTok{cor}\NormalTok{(homedata}\SpecialCharTok{$}\NormalTok{y1970, homedata}\SpecialCharTok{$}\NormalTok{y2000)}
\end{Highlighting}
\end{Shaded}

\begin{Shaded}
\begin{Highlighting}[]
\CommentTok{\#4.}
\CommentTok{\#View(anscombe)}

\CommentTok{\#boxplot(anscombe)}
\end{Highlighting}
\end{Shaded}

От упражнение 4

\begin{Shaded}
\begin{Highlighting}[]
\NormalTok{dice }\OtherTok{=} \ControlFlowTok{function}\NormalTok{(}\AttributeTok{N =} \DecValTok{100}\NormalTok{)\{}
\NormalTok{  samples }\OtherTok{\textless{}{-}} \FunctionTok{sample}\NormalTok{(}\DecValTok{1}\SpecialCharTok{:}\DecValTok{6}\NormalTok{, }\AttributeTok{size =} \DecValTok{100}\NormalTok{, }\AttributeTok{replace =} \ConstantTok{TRUE}\NormalTok{)}
  
\NormalTok{  result }\OtherTok{\textless{}{-}} \FunctionTok{sum}\NormalTok{(samples }\SpecialCharTok{==} \DecValTok{6}\NormalTok{)}
  
\NormalTok{  result}
\NormalTok{\}}

\CommentTok{\#емпирична вероятност}
\FunctionTok{dice}\NormalTok{() }\SpecialCharTok{/} \DecValTok{100}
\end{Highlighting}
\end{Shaded}

\begin{verbatim}
## [1] 0.13
\end{verbatim}

\begin{Shaded}
\begin{Highlighting}[]
\NormalTok{birthdays }\OtherTok{=} \ControlFlowTok{function}\NormalTok{(}\AttributeTok{p =} \FloatTok{0.5}\NormalTok{)\{}
  
\NormalTok{  prob }\OtherTok{=} \DecValTok{1}
  \ControlFlowTok{for}\NormalTok{(i }\ControlFlowTok{in} \DecValTok{1}\SpecialCharTok{:}\DecValTok{365}\NormalTok{)\{}
\NormalTok{    prob }\OtherTok{=}\NormalTok{ prob }\SpecialCharTok{*}\NormalTok{ (}\DecValTok{366} \SpecialCharTok{{-}}\NormalTok{ i) }\SpecialCharTok{/} \DecValTok{365}
    
    \ControlFlowTok{if}\NormalTok{(prob }\SpecialCharTok{\textless{}}  \DecValTok{1} \SpecialCharTok{{-}}\NormalTok{ p) }\ControlFlowTok{break}
\NormalTok{  \}}
      \FunctionTok{return}\NormalTok{(i)}
\NormalTok{\}}

\FunctionTok{birthdays}\NormalTok{()}
\end{Highlighting}
\end{Shaded}

\begin{verbatim}
## [1] 23
\end{verbatim}

\begin{Shaded}
\begin{Highlighting}[]
\NormalTok{game\_one }\OtherTok{=} \ControlFlowTok{function}\NormalTok{(father, mother)\{}
  
\NormalTok{  wins }\OtherTok{=} \DecValTok{0}
  
  \ControlFlowTok{for}\NormalTok{(i }\ControlFlowTok{in} \DecValTok{1}\SpecialCharTok{:}\DecValTok{1000}\NormalTok{)\{}
\NormalTok{      vs\_mom }\OtherTok{\textless{}{-}} \FunctionTok{sample}\NormalTok{(}\DecValTok{0}\SpecialCharTok{:}\DecValTok{1}\NormalTok{, }\DecValTok{2}\NormalTok{, }\AttributeTok{replace =}\NormalTok{ T, }\AttributeTok{prob =} \FunctionTok{c}\NormalTok{(}\DecValTok{1} \SpecialCharTok{{-}}\NormalTok{ mother, mother))}
\NormalTok{      vs\_dad }\OtherTok{\textless{}{-}} \FunctionTok{sample}\NormalTok{(}\DecValTok{0}\SpecialCharTok{:}\DecValTok{1}\NormalTok{, }\DecValTok{1}\NormalTok{, }\AttributeTok{replace =}\NormalTok{ T, }\AttributeTok{prob =} \FunctionTok{c}\NormalTok{(}\DecValTok{1}\SpecialCharTok{{-}}\NormalTok{ father, father))}
  
      \ControlFlowTok{if}\NormalTok{(vs\_mom[}\DecValTok{1}\NormalTok{] }\SpecialCharTok{==} \DecValTok{1} \SpecialCharTok{\&}\NormalTok{ vs\_dad }\SpecialCharTok{==} \DecValTok{1} \SpecialCharTok{|}\NormalTok{ vs\_mom[}\DecValTok{2}\NormalTok{] }\SpecialCharTok{==} \DecValTok{1} \SpecialCharTok{\&}\NormalTok{ vs\_dad }\SpecialCharTok{==} \DecValTok{1}\NormalTok{)\{}
\NormalTok{        wins }\OtherTok{=}\NormalTok{ wins }\SpecialCharTok{+} \DecValTok{1}
\NormalTok{      \}}
      
\NormalTok{  \}}
  \FunctionTok{return}\NormalTok{(wins}\SpecialCharTok{/}\DecValTok{1000}\NormalTok{)}
\NormalTok{\}}

\FunctionTok{game\_one}\NormalTok{(}\FloatTok{0.3}\NormalTok{, }\FloatTok{0.4}\NormalTok{)}
\end{Highlighting}
\end{Shaded}

\begin{verbatim}
## [1] 0.214
\end{verbatim}

\begin{Shaded}
\begin{Highlighting}[]
\CommentTok{\#4.}

\NormalTok{presents }\OtherTok{=} \ControlFlowTok{function}\NormalTok{(}\AttributeTok{n =} \DecValTok{20}\NormalTok{)\{}
  
  \ControlFlowTok{for}\NormalTok{(j }\ControlFlowTok{in} \DecValTok{1}\SpecialCharTok{:}\DecValTok{10000}\NormalTok{)\{}
\NormalTok{      counter }\OtherTok{=} \DecValTok{1}
  
\NormalTok{       x }\OtherTok{\textless{}{-}} \FunctionTok{sample}\NormalTok{(}\DecValTok{1}\SpecialCharTok{:}\NormalTok{n, n, }\AttributeTok{replace =} \ConstantTok{FALSE}\NormalTok{)}
  
      \ControlFlowTok{for}\NormalTok{(i }\ControlFlowTok{in} \DecValTok{1}\SpecialCharTok{:}\DecValTok{20}\NormalTok{)\{}
    
      \ControlFlowTok{if}\NormalTok{(i }\SpecialCharTok{==}\NormalTok{ x[i])\{}
\NormalTok{        counter }\OtherTok{=}\NormalTok{ counter }\SpecialCharTok{+} \DecValTok{1}
        \ControlFlowTok{break}
\NormalTok{        \}}
\NormalTok{      \}}
      \FunctionTok{return}\NormalTok{(n }\SpecialCharTok{{-}}\NormalTok{ counter)}
\NormalTok{  \}}
\NormalTok{\}}

\FunctionTok{presents}\NormalTok{() }\SpecialCharTok{/} \DecValTok{10000}
\end{Highlighting}
\end{Shaded}

\begin{verbatim}
## [1] 0.0019
\end{verbatim}

\begin{Shaded}
\begin{Highlighting}[]
\CommentTok{\#5.}

\NormalTok{coins }\OtherTok{=} \ControlFlowTok{function}\NormalTok{()\{}
  \ControlFlowTok{for}\NormalTok{(i }\ControlFlowTok{in} \DecValTok{1}\SpecialCharTok{:}\DecValTok{10000}\NormalTok{)\{}
    
\NormalTok{    x }\OtherTok{\textless{}{-}} \FunctionTok{sample}\NormalTok{(}\DecValTok{0}\SpecialCharTok{:}\DecValTok{1}\NormalTok{, }\DecValTok{5}\NormalTok{, }\AttributeTok{replace =}\NormalTok{ T)}
    
    \ControlFlowTok{if}\NormalTok{(x[}\DecValTok{1}\NormalTok{] }\SpecialCharTok{==} \DecValTok{1} \SpecialCharTok{\&}\NormalTok{ x[}\DecValTok{2}\NormalTok{] }\SpecialCharTok{==} \DecValTok{1} \SpecialCharTok{\&}\NormalTok{ x[}\DecValTok{3}\NormalTok{] }\SpecialCharTok{==} \DecValTok{0} \SpecialCharTok{\&}\NormalTok{ x[}\DecValTok{4}\NormalTok{] }\SpecialCharTok{==} \DecValTok{1} \SpecialCharTok{\&}\NormalTok{ x[}\DecValTok{5}\NormalTok{] }\SpecialCharTok{==} \DecValTok{0}\NormalTok{)\{}
      \ControlFlowTok{break}
\NormalTok{    \}}
\NormalTok{  \}}
  \FunctionTok{return}\NormalTok{(i)}
\NormalTok{\}}

\FunctionTok{coins}\NormalTok{()}
\end{Highlighting}
\end{Shaded}

\begin{verbatim}
## [1] 28
\end{verbatim}

От упражнение 5

\begin{Shaded}
\begin{Highlighting}[]
\CommentTok{\#вероятността да се паднат по{-}малко от 5 шестици при хвърляне на 30 зара}
\FunctionTok{pbinom}\NormalTok{(}\AttributeTok{q =} \DecValTok{4}\NormalTok{, }\AttributeTok{size =} \DecValTok{30}\NormalTok{, }\AttributeTok{prob =} \DecValTok{1}\SpecialCharTok{/}\DecValTok{6}\NormalTok{) }
\end{Highlighting}
\end{Shaded}

\begin{verbatim}
## [1] 0.4243389
\end{verbatim}

\begin{Shaded}
\begin{Highlighting}[]
\CommentTok{\#взимаме извадка от 10000 по 30 хвърляния на зар и го правим на таблица {-} това е емп. вер.}
\NormalTok{thrown\_dices }\OtherTok{\textless{}{-}} \FunctionTok{rbinom}\NormalTok{(}\AttributeTok{n =} \DecValTok{10000}\NormalTok{, }\AttributeTok{size =} \DecValTok{30}\NormalTok{, }\AttributeTok{prob =} \DecValTok{1}\SpecialCharTok{/}\DecValTok{6}\NormalTok{)}

\NormalTok{thrown\_dices }\SpecialCharTok{\%\textgreater{}\%} \FunctionTok{table}\NormalTok{() }\SpecialCharTok{\%\textgreater{}\%} \FunctionTok{prop.table}\NormalTok{()}
\end{Highlighting}
\end{Shaded}

\begin{verbatim}
## .
##      0      1      2      3      4      5      6      7      8      9     10 
## 0.0043 0.0242 0.0717 0.1355 0.1842 0.1894 0.1616 0.1104 0.0667 0.0341 0.0115 
##     11     12     13 
## 0.0043 0.0017 0.0004
\end{verbatim}

\begin{Shaded}
\begin{Highlighting}[]
\CommentTok{\#това е теоритичната вероятност}
\FunctionTok{dbinom}\NormalTok{(}\DecValTok{0}\SpecialCharTok{:}\DecValTok{6}\NormalTok{, }\AttributeTok{size =} \DecValTok{30}\NormalTok{, }\AttributeTok{prob =} \DecValTok{1}\SpecialCharTok{/}\DecValTok{6}\NormalTok{)}
\end{Highlighting}
\end{Shaded}

\begin{verbatim}
## [1] 0.00421272 0.02527632 0.07330133 0.13682915 0.18471936 0.19210813 0.16009011
\end{verbatim}

\begin{Shaded}
\begin{Highlighting}[]
\CommentTok{\#с вероятност 0,75 да се паднат повече от колко шестици}
\CommentTok{\#понеже нямаме ф{-}я за повече от ние ще променим твърдението с неговото обратно}
\CommentTok{\#понеже qbinom показва колко най{-}много шестици ще се паднат за някаква вероятност}
\CommentTok{\#тоест ние го променяме колко най{-}много ще се паднат за 0.25 вероятност}
\FunctionTok{qbinom}\NormalTok{(}\AttributeTok{p =} \FloatTok{0.25}\NormalTok{, }\AttributeTok{size =} \DecValTok{30}\NormalTok{, }\AttributeTok{prob =} \DecValTok{1}\SpecialCharTok{/}\DecValTok{6}\NormalTok{)}
\end{Highlighting}
\end{Shaded}

\begin{verbatim}
## [1] 4
\end{verbatim}

\begin{Shaded}
\begin{Highlighting}[]
\FunctionTok{qbinom}\NormalTok{(}\AttributeTok{p =} \FloatTok{0.75}\NormalTok{, }\AttributeTok{size =} \DecValTok{30}\NormalTok{, }\AttributeTok{prob =} \DecValTok{1}\SpecialCharTok{/}\DecValTok{6}\NormalTok{, }\AttributeTok{lower.tail =} \ConstantTok{FALSE}\NormalTok{)}
\end{Highlighting}
\end{Shaded}

\begin{verbatim}
## [1] 4
\end{verbatim}

\begin{Shaded}
\begin{Highlighting}[]
\CommentTok{\#2.}

\CommentTok{\#имаме пет неуспеха преди 3тия успех като вероятността за успех е 0.2}
\CommentTok{\# x е квантил}
\FunctionTok{dnbinom}\NormalTok{(}\AttributeTok{x =} \DecValTok{5}\NormalTok{, }\AttributeTok{size =} \DecValTok{3}\NormalTok{, }\AttributeTok{prob =} \FloatTok{0.2}\NormalTok{)}
\end{Highlighting}
\end{Shaded}

\begin{verbatim}
## [1] 0.05505024
\end{verbatim}

\begin{Shaded}
\begin{Highlighting}[]
\CommentTok{\#вероятност да са му нужни повече от 6 изтрела}
\CommentTok{\#q е бр. неуспехи}
\FunctionTok{pnbinom}\NormalTok{(}\AttributeTok{q =} \DecValTok{3}\NormalTok{, }\AttributeTok{size =} \DecValTok{3}\NormalTok{, }\AttributeTok{prob =} \FloatTok{0.2}\NormalTok{, }\AttributeTok{lower.tail =} \ConstantTok{FALSE}\NormalTok{)}
\end{Highlighting}
\end{Shaded}

\begin{verbatim}
## [1] 0.90112
\end{verbatim}

\begin{Shaded}
\begin{Highlighting}[]
\CommentTok{\#вероятност да му трябват между 5 и 8 изтрела вкл.}
\FunctionTok{pnbinom}\NormalTok{(}\AttributeTok{q =} \DecValTok{5}\NormalTok{, }\AttributeTok{size =} \DecValTok{3}\NormalTok{, }\AttributeTok{prob =} \FloatTok{0.2}\NormalTok{) }\SpecialCharTok{{-}} \FunctionTok{pnbinom}\NormalTok{(}\DecValTok{1}\NormalTok{,}\DecValTok{3}\NormalTok{,}\FloatTok{0.2}\NormalTok{)}
\end{Highlighting}
\end{Shaded}

\begin{verbatim}
## [1] 0.1758822
\end{verbatim}

\begin{Shaded}
\begin{Highlighting}[]
\CommentTok{\#3.}
\NormalTok{balls }\OtherTok{=} \ControlFlowTok{function}\NormalTok{()\{}
  
\NormalTok{  total }\OtherTok{=} \DecValTok{13}
\NormalTok{  white\_balls }\OtherTok{=} \DecValTok{7}
\NormalTok{  black\_balls }\OtherTok{=} \DecValTok{6}
  
\NormalTok{  num\_white }\OtherTok{=} \DecValTok{0}
  
  \ControlFlowTok{for}\NormalTok{(i }\ControlFlowTok{in} \DecValTok{1}\SpecialCharTok{:}\DecValTok{8}\NormalTok{)\{}
    
\NormalTok{    white\_p }\OtherTok{=}\NormalTok{ white\_balls }\SpecialCharTok{/}\NormalTok{ total}
\NormalTok{    black\_p }\OtherTok{=}\NormalTok{ black\_balls }\SpecialCharTok{/}\NormalTok{ total}
    
\NormalTok{    x }\OtherTok{\textless{}{-}} \FunctionTok{sample}\NormalTok{(}\FunctionTok{c}\NormalTok{(}\DecValTok{0}\NormalTok{,}\DecValTok{1}\NormalTok{), }\AttributeTok{size =} \DecValTok{1}\NormalTok{, }\AttributeTok{prob =} \FunctionTok{c}\NormalTok{(black\_p, white\_p))}
    
    \ControlFlowTok{if}\NormalTok{(x }\SpecialCharTok{==} \DecValTok{0}\NormalTok{)\{}
\NormalTok{      black\_balls }\OtherTok{=}\NormalTok{ black\_balls }\SpecialCharTok{{-}} \DecValTok{1}
\NormalTok{    \}}
    \ControlFlowTok{else}\NormalTok{\{}
\NormalTok{      white\_balls }\OtherTok{=}\NormalTok{ white\_balls }\SpecialCharTok{{-}} \DecValTok{1}
\NormalTok{      num\_white }\OtherTok{=}\NormalTok{ num\_white }\SpecialCharTok{+} \DecValTok{1}
\NormalTok{    \}}
    
\NormalTok{    total }\OtherTok{=}\NormalTok{ total }\SpecialCharTok{{-}} \DecValTok{1}
\NormalTok{  \}}
\NormalTok{  num\_white}
\NormalTok{\}}

\NormalTok{replicated }\OtherTok{\textless{}{-}} \FunctionTok{replicate}\NormalTok{(}\DecValTok{1000}\NormalTok{, }\FunctionTok{balls}\NormalTok{())}


\FunctionTok{mean}\NormalTok{(replicated)}
\end{Highlighting}
\end{Shaded}

\begin{verbatim}
## [1] 4.363
\end{verbatim}

\begin{Shaded}
\begin{Highlighting}[]
\FunctionTok{sd}\NormalTok{(replicated)}
\end{Highlighting}
\end{Shaded}

\begin{verbatim}
## [1] 0.9295672
\end{verbatim}

\begin{Shaded}
\begin{Highlighting}[]
\FunctionTok{min}\NormalTok{(replicated)}
\end{Highlighting}
\end{Shaded}

\begin{verbatim}
## [1] 2
\end{verbatim}

\begin{Shaded}
\begin{Highlighting}[]
\FunctionTok{max}\NormalTok{(replicated)}
\end{Highlighting}
\end{Shaded}

\begin{verbatim}
## [1] 7
\end{verbatim}

\begin{Shaded}
\begin{Highlighting}[]
\FunctionTok{sum}\NormalTok{(replicated }\SpecialCharTok{==} \DecValTok{3}\NormalTok{) }\SpecialCharTok{/} \DecValTok{1000}
\end{Highlighting}
\end{Shaded}

\begin{verbatim}
## [1] 0.162
\end{verbatim}

\begin{Shaded}
\begin{Highlighting}[]
\CommentTok{\#емпирична вероятност}
\NormalTok{emp }\OtherTok{\textless{}{-}}\NormalTok{ replicated }\SpecialCharTok{\%\textgreater{}\%} \FunctionTok{table}\NormalTok{() }\SpecialCharTok{\%\textgreater{}\%} \FunctionTok{prop.table}\NormalTok{()}

\CommentTok{\#теоритична вероятност}
\NormalTok{theor }\OtherTok{\textless{}{-}} \FunctionTok{dhyper}\NormalTok{(}\DecValTok{0}\SpecialCharTok{:}\DecValTok{8}\NormalTok{, }\DecValTok{7}\NormalTok{, }\DecValTok{6}\NormalTok{, }\DecValTok{8}\NormalTok{)}

\FunctionTok{hist}\NormalTok{(}\FunctionTok{c}\NormalTok{(emp, theor) , }\AttributeTok{beside =}\NormalTok{ T)}
\end{Highlighting}
\end{Shaded}

\begin{verbatim}
## Warning in plot.window(xlim, ylim, "", ...): "beside" is not a graphical
## parameter
\end{verbatim}

\begin{verbatim}
## Warning in title(main = main, sub = sub, xlab = xlab, ylab = ylab, ...):
## "beside" is not a graphical parameter
\end{verbatim}

\begin{verbatim}
## Warning in axis(1, ...): "beside" is not a graphical parameter
\end{verbatim}

\begin{verbatim}
## Warning in axis(2, ...): "beside" is not a graphical parameter
\end{verbatim}

\includegraphics{examPrep_files/figure-latex/unnamed-chunk-19-1.pdf}

\begin{Shaded}
\begin{Highlighting}[]
\CommentTok{\#4.}
\NormalTok{n }\OtherTok{=} \DecValTok{100}

\FunctionTok{dbinom}\NormalTok{(}\DecValTok{2}\NormalTok{, }\AttributeTok{size =}\NormalTok{ n, }\AttributeTok{prob =} \DecValTok{5}\SpecialCharTok{/}\NormalTok{(}\DecValTok{2}\SpecialCharTok{*}\NormalTok{n))}
\end{Highlighting}
\end{Shaded}

\begin{verbatim}
## [1] 0.2587841
\end{verbatim}

От упражнение 6

\begin{Shaded}
\begin{Highlighting}[]
\CommentTok{\#нормално разпр с боксплот и хистограма}
\NormalTok{normal }\OtherTok{\textless{}{-}} \FunctionTok{rnorm}\NormalTok{(}\DecValTok{100}\NormalTok{, }\DecValTok{5}\NormalTok{, }\FunctionTok{sqrt}\NormalTok{(}\DecValTok{2}\NormalTok{))}

\FunctionTok{boxplot}\NormalTok{(normal)}
\end{Highlighting}
\end{Shaded}

\includegraphics{examPrep_files/figure-latex/unnamed-chunk-21-1.pdf}

\begin{Shaded}
\begin{Highlighting}[]
\FunctionTok{hist}\NormalTok{(normal)}
\end{Highlighting}
\end{Shaded}

\includegraphics{examPrep_files/figure-latex/unnamed-chunk-21-2.pdf}

\begin{Shaded}
\begin{Highlighting}[]
\CommentTok{\#правим някаква извадка}
\NormalTok{  s }\OtherTok{\textless{}{-}} \FunctionTok{seq}\NormalTok{(}\DecValTok{1}\NormalTok{, }\DecValTok{8}\NormalTok{, }\FloatTok{0.2}\NormalTok{)}
  
\CommentTok{\#теоритична вероятност  }
\FunctionTok{dnorm}\NormalTok{(s, }\DecValTok{5}\NormalTok{, }\FunctionTok{sqrt}\NormalTok{(}\DecValTok{2}\NormalTok{))}
\end{Highlighting}
\end{Shaded}

\begin{verbatim}
##  [1] 0.005166746 0.007631185 0.011047931 0.015677760 0.021807265 0.029732572
##  [7] 0.039735427 0.052051997 0.066836087 0.084119899 0.103776874 0.125492144
## [13] 0.148746447 0.172818715 0.196810858 0.219695645 0.240385325 0.257815227
## [19] 0.271033697 0.279287902 0.282094792 0.279287902 0.271033697 0.257815227
## [25] 0.240385325 0.219695645 0.196810858 0.172818715 0.148746447 0.125492144
## [31] 0.103776874 0.084119899 0.066836087 0.052051997 0.039735427 0.029732572
\end{verbatim}

\begin{Shaded}
\begin{Highlighting}[]
\CommentTok{\#2.}
\CommentTok{\#n {-} брой сл. в}
\CommentTok{\#k {-} брой стойности, които ни дава всяко разпределение}
\CommentTok{\# fn {-} distribution function and ... is her arguments}

\NormalTok{xsim }\OtherTok{\textless{}{-}} \ControlFlowTok{function}\NormalTok{(n, k, fn, ...)\{}
  
  \CommentTok{\#вектор пълен с нули. В него се събират стойностите поиндексно на всяко разпределение}
\NormalTok{  s }\OtherTok{\textless{}{-}} \FunctionTok{rep}\NormalTok{(}\DecValTok{0}\NormalTok{, k)}
  \ControlFlowTok{for}\NormalTok{ (i }\ControlFlowTok{in} \DecValTok{1}\SpecialCharTok{:}\NormalTok{n) \{}
\NormalTok{    s }\OtherTok{\textless{}{-}}\NormalTok{ s }\SpecialCharTok{+} \FunctionTok{fn}\NormalTok{(k, ... )}
\NormalTok{  \}}
  \CommentTok{\#връща се вектор със сумата поиндексно на всички разпределения}
\NormalTok{  s}
\NormalTok{\}}

\FunctionTok{xsim}\NormalTok{(}\DecValTok{100}\NormalTok{, }\DecValTok{100}\NormalTok{, rexp) }\SpecialCharTok{\%\textgreater{}\%} \FunctionTok{hist}\NormalTok{()}
\end{Highlighting}
\end{Shaded}

\includegraphics{examPrep_files/figure-latex/unnamed-chunk-22-1.pdf}

\begin{Shaded}
\begin{Highlighting}[]
\CommentTok{\#това ни показва граничната теор {-} т.е при сумиране на независими еднакво разпр сл.в}
\CommentTok{\#че се получава нормално разпределение}
\end{Highlighting}
\end{Shaded}

\begin{Shaded}
\begin{Highlighting}[]
\CommentTok{\#4. }
\CommentTok{\#пъпеши по{-}малки от 20 т.е трето качество}
\NormalTok{small }\OtherTok{\textless{}{-}} \FunctionTok{pnorm}\NormalTok{(}\DecValTok{20}\NormalTok{, }\DecValTok{25}\NormalTok{, }\DecValTok{6}\NormalTok{)}

\CommentTok{\#първата половина от по{-}големите}
\NormalTok{medium }\OtherTok{\textless{}{-}}\NormalTok{ (}\DecValTok{1} \SpecialCharTok{{-}}\NormalTok{ small) }\SpecialCharTok{/} \DecValTok{2}

\NormalTok{big }\OtherTok{\textless{}{-}}\NormalTok{ medium}

\CommentTok{\#колко да е голям за да бъде трето качество }
\FunctionTok{qnorm}\NormalTok{(big }\SpecialCharTok{+}\NormalTok{ medium, }\DecValTok{25}\NormalTok{, }\DecValTok{6}\NormalTok{)}
\end{Highlighting}
\end{Shaded}

\begin{verbatim}
## [1] 30
\end{verbatim}

От упражнение 7

\begin{Shaded}
\begin{Highlighting}[]
\CommentTok{\#1. a)ако ни е известно станд отклонение}
\NormalTok{n }\OtherTok{\textless{}{-}} \DecValTok{20}
\NormalTok{sd }\OtherTok{\textless{}{-}} \DecValTok{2}

\NormalTok{x }\OtherTok{\textless{}{-}} \FunctionTok{rnorm}\NormalTok{(n, }\DecValTok{3}\NormalTok{, sd)}

\NormalTok{q }\OtherTok{\textless{}{-}} \FunctionTok{qnorm}\NormalTok{(}\FloatTok{0.975}\NormalTok{, }\DecValTok{3}\NormalTok{, }\DecValTok{2}\NormalTok{)}

\NormalTok{left\_interval }\OtherTok{\textless{}{-}} \FunctionTok{mean}\NormalTok{(x) }\SpecialCharTok{{-}}\NormalTok{ q }\SpecialCharTok{*}\NormalTok{ (sd }\SpecialCharTok{/} \FunctionTok{sqrt}\NormalTok{(n))}

\NormalTok{right\_interval }\OtherTok{\textless{}{-}} \FunctionTok{mean}\NormalTok{(x) }\SpecialCharTok{+}\NormalTok{ q }\SpecialCharTok{*}\NormalTok{ (sd }\SpecialCharTok{/} \FunctionTok{sqrt}\NormalTok{(n))}

\CommentTok{\#b)ако не ни е известно станд отклонение}

\NormalTok{n }\OtherTok{\textless{}{-}} \DecValTok{20}

\NormalTok{theor\_mean }\OtherTok{\textless{}{-}} \DecValTok{3}
\NormalTok{theor\_sd }\OtherTok{\textless{}{-}} \DecValTok{2}

\NormalTok{x }\OtherTok{\textless{}{-}} \FunctionTok{rnorm}\NormalTok{(n, theor\_mean, theor\_sd)}

\NormalTok{m }\OtherTok{\textless{}{-}} \FunctionTok{mean}\NormalTok{(x)}

\NormalTok{sd }\OtherTok{\textless{}{-}} \FunctionTok{sd}\NormalTok{(x)}

\NormalTok{q }\OtherTok{\textless{}{-}} \FunctionTok{qt}\NormalTok{(}\AttributeTok{p =} \FloatTok{0.975}\NormalTok{, }\AttributeTok{df =}\NormalTok{ n }\SpecialCharTok{{-}} \DecValTok{1}\NormalTok{)}

\NormalTok{left\_interval }\OtherTok{\textless{}{-}}\NormalTok{ m }\SpecialCharTok{{-}}\NormalTok{ q }\SpecialCharTok{*}\NormalTok{ sd }\SpecialCharTok{/} \FunctionTok{sqrt}\NormalTok{(n)}

\NormalTok{right\_interval }\OtherTok{\textless{}{-}}\NormalTok{ m }\SpecialCharTok{+}\NormalTok{ q }\SpecialCharTok{*}\NormalTok{ sd }\SpecialCharTok{/} \FunctionTok{sqrt}\NormalTok{(n)}

\CommentTok{\#можем и така да намерим доверителния интервал ако знаем че х е нормално разпределено}
\FunctionTok{t.test}\NormalTok{(x)}
\end{Highlighting}
\end{Shaded}

\begin{verbatim}
## 
##  One Sample t-test
## 
## data:  x
## t = 5.0664, df = 19, p-value = 6.856e-05
## alternative hypothesis: true mean is not equal to 0
## 95 percent confidence interval:
##  0.9308933 2.2414601
## sample estimates:
## mean of x 
##  1.586177
\end{verbatim}

\begin{Shaded}
\begin{Highlighting}[]
\CommentTok{\#2.}
\NormalTok{data1 }\OtherTok{\textless{}{-}} \FunctionTok{c}\NormalTok{(}\FloatTok{10.0}\NormalTok{, }\FloatTok{13.6}\NormalTok{, }\FloatTok{13.2}\NormalTok{, }\FloatTok{11.6}\NormalTok{, }\FloatTok{12.5}\NormalTok{, }\FloatTok{14.2}\NormalTok{, }\FloatTok{14.9}\NormalTok{, }\FloatTok{14.5}\NormalTok{, }\FloatTok{13.4}\NormalTok{, }\FloatTok{8.6}\NormalTok{, }\FloatTok{11.5}\NormalTok{, }\FloatTok{16.0}\NormalTok{, }\FloatTok{14.2}\NormalTok{, }\FloatTok{16.8}\NormalTok{, }\FloatTok{17.9}\NormalTok{, }\FloatTok{17.0}\NormalTok{)}

\CommentTok{\#проверяваме дали е нормално разпределена}
\FunctionTok{qqnorm}\NormalTok{(data1)}
\end{Highlighting}
\end{Shaded}

\includegraphics{examPrep_files/figure-latex/unnamed-chunk-25-1.pdf}

\begin{Shaded}
\begin{Highlighting}[]
\CommentTok{\#правим доверителните интервали}
\FunctionTok{t.test}\NormalTok{(data1)}
\end{Highlighting}
\end{Shaded}

\begin{verbatim}
## 
##  One Sample t-test
## 
## data:  data1
## t = 21.65, df = 15, p-value = 9.976e-13
## alternative hypothesis: true mean is not equal to 0
## 95 percent confidence interval:
##  12.39066 15.09684
## sample estimates:
## mean of x 
##  13.74375
\end{verbatim}

\begin{Shaded}
\begin{Highlighting}[]
\FunctionTok{t.test}\NormalTok{(data1, }\AttributeTok{conf.level =} \FloatTok{0.90}\NormalTok{)}
\end{Highlighting}
\end{Shaded}

\begin{verbatim}
## 
##  One Sample t-test
## 
## data:  data1
## t = 21.65, df = 15, p-value = 9.976e-13
## alternative hypothesis: true mean is not equal to 0
## 90 percent confidence interval:
##  12.63088 14.85662
## sample estimates:
## mean of x 
##  13.74375
\end{verbatim}

\begin{Shaded}
\begin{Highlighting}[]
\CommentTok{\#3.a)}
\FunctionTok{qqnorm}\NormalTok{(rat)}
\FunctionTok{qqline}\NormalTok{(rat)}
\end{Highlighting}
\end{Shaded}

\includegraphics{examPrep_files/figure-latex/unnamed-chunk-26-1.pdf}

\begin{Shaded}
\begin{Highlighting}[]
\FunctionTok{t.test}\NormalTok{(rat, }\AttributeTok{conf.level =} \FloatTok{0.96}\NormalTok{)}
\end{Highlighting}
\end{Shaded}

\begin{verbatim}
## 
##  One Sample t-test
## 
## data:  rat
## t = 14.176, df = 19, p-value = 1.48e-11
## alternative hypothesis: true mean is not equal to 0
## 96 percent confidence interval:
##   95.80624 131.09376
## sample estimates:
## mean of x 
##    113.45
\end{verbatim}

\begin{Shaded}
\begin{Highlighting}[]
\CommentTok{\#b)}
\CommentTok{\#данните не са нормално разпределени затова ползваме wilcox.test}
\FunctionTok{qqnorm}\NormalTok{(exec.pay)}
\FunctionTok{qqline}\NormalTok{(exec.pay)}
\end{Highlighting}
\end{Shaded}

\includegraphics{examPrep_files/figure-latex/unnamed-chunk-26-2.pdf}

\begin{Shaded}
\begin{Highlighting}[]
\FunctionTok{wilcox.test}\NormalTok{(exec.pay, }\AttributeTok{conf.int =}\NormalTok{ T, }\AttributeTok{conf.level =} \FloatTok{0.96}\NormalTok{)}
\end{Highlighting}
\end{Shaded}

\begin{verbatim}
## 
##  Wilcoxon signed rank test with continuity correction
## 
## data:  exec.pay
## V = 19306, p-value < 2.2e-16
## alternative hypothesis: true location is not equal to 0
## 96 percent confidence interval:
##  25.99996 33.00003
## sample estimates:
## (pseudo)median 
##       29.00002
\end{verbatim}

\begin{Shaded}
\begin{Highlighting}[]
\CommentTok{\#v)}
\end{Highlighting}
\end{Shaded}

\begin{Shaded}
\begin{Highlighting}[]
\CommentTok{\#4.когато имаме пропорции ползваме това}
\FunctionTok{prop.test}\NormalTok{(}\DecValTok{87}\NormalTok{, }\DecValTok{150}\NormalTok{, }\AttributeTok{conf.level =} \FloatTok{0.92}\NormalTok{)}
\end{Highlighting}
\end{Shaded}

\begin{verbatim}
## 
##  1-sample proportions test with continuity correction
## 
## data:  87 out of 150, null probability 0.5
## X-squared = 3.5267, df = 1, p-value = 0.06039
## alternative hypothesis: true p is not equal to 0.5
## 92 percent confidence interval:
##  0.5051991 0.6514474
## sample estimates:
##    p 
## 0.58
\end{verbatim}

\begin{Shaded}
\begin{Highlighting}[]
\CommentTok{\#6.}
\NormalTok{smoke\_men }\OtherTok{\textless{}{-}} \FunctionTok{nrow}\NormalTok{(survey[survey}\SpecialCharTok{$}\NormalTok{Sex }\SpecialCharTok{==} \StringTok{\textquotesingle{}Male\textquotesingle{}} \SpecialCharTok{\&}\NormalTok{ survey}\SpecialCharTok{$}\NormalTok{Smoke }\SpecialCharTok{==} \StringTok{\textquotesingle{}Never\textquotesingle{}}\NormalTok{, ])}

\NormalTok{all\_men }\OtherTok{\textless{}{-}} \FunctionTok{nrow}\NormalTok{(survey[survey}\SpecialCharTok{$}\NormalTok{Sex }\SpecialCharTok{==} \StringTok{\textquotesingle{}Male\textquotesingle{}}\NormalTok{, ])}

\FunctionTok{prop.test}\NormalTok{(smoke\_men, all\_men, }\AttributeTok{conf.level =} \FloatTok{0.90}\NormalTok{)}
\end{Highlighting}
\end{Shaded}

\begin{verbatim}
## 
##  1-sample proportions test with continuity correction
## 
## data:  smoke_men out of all_men, null probability 0.5
## X-squared = 32.303, df = 1, p-value = 1.319e-08
## alternative hypothesis: true p is not equal to 0.5
## 90 percent confidence interval:
##  0.6908187 0.8260629
## sample estimates:
##         p 
## 0.7647059
\end{verbatim}

От упражнение 8

\begin{Shaded}
\begin{Highlighting}[]
\CommentTok{\#1.check for norm distr}
\CommentTok{\# H0 {-} EX = 3}
\NormalTok{data2 }\OtherTok{\textless{}{-}} \FunctionTok{rnorm}\NormalTok{(}\DecValTok{100}\NormalTok{, }\AttributeTok{mean =} \DecValTok{2}\NormalTok{, }\AttributeTok{sd =} \DecValTok{2}\NormalTok{)}

\CommentTok{\#p{-}value\textless{} 0.05 отхвърляме хипотезата}
\FunctionTok{t.test}\NormalTok{(data2, }\AttributeTok{mu =} \DecValTok{3}\NormalTok{, }\AttributeTok{alternative =} \StringTok{\textquotesingle{}two.sided\textquotesingle{}}\NormalTok{)}
\end{Highlighting}
\end{Shaded}

\begin{verbatim}
## 
##  One Sample t-test
## 
## data:  data2
## t = -6.184, df = 99, p-value = 1.407e-08
## alternative hypothesis: true mean is not equal to 3
## 95 percent confidence interval:
##  1.473201 2.214979
## sample estimates:
## mean of x 
##   1.84409
\end{verbatim}

\begin{Shaded}
\begin{Highlighting}[]
\FunctionTok{t.test}\NormalTok{(data2, }\AttributeTok{mu =} \DecValTok{5}\NormalTok{, }\AttributeTok{alternative =} \StringTok{\textquotesingle{}two.sided\textquotesingle{}}\NormalTok{)}
\end{Highlighting}
\end{Shaded}

\begin{verbatim}
## 
##  One Sample t-test
## 
## data:  data2
## t = -16.884, df = 99, p-value < 2.2e-16
## alternative hypothesis: true mean is not equal to 5
## 95 percent confidence interval:
##  1.473201 2.214979
## sample estimates:
## mean of x 
##   1.84409
\end{verbatim}

\begin{Shaded}
\begin{Highlighting}[]
\CommentTok{\#2.H0: дните за почивка да са 24 при п{-}жалуе \textgreater{} 0.2}
\NormalTok{data3 }\OtherTok{\textless{}{-}}\NormalTok{ vacation}

\CommentTok{\#seems normal distributed}
\FunctionTok{qqnorm}\NormalTok{(data3)}
\FunctionTok{qqline}\NormalTok{(data3)}
\end{Highlighting}
\end{Shaded}

\includegraphics{examPrep_files/figure-latex/unnamed-chunk-30-1.pdf}

\begin{Shaded}
\begin{Highlighting}[]
\CommentTok{\#p{-}value \textless{} 0,2 отхвърляме нулевата хипотеза}
\FunctionTok{t.test}\NormalTok{(vacation, }\AttributeTok{mu =} \DecValTok{24}\NormalTok{, }\AttributeTok{alternative =} \StringTok{\textquotesingle{}two.sided\textquotesingle{}}\NormalTok{)}
\end{Highlighting}
\end{Shaded}

\begin{verbatim}
## 
##  One Sample t-test
## 
## data:  vacation
## t = -2.2584, df = 34, p-value = 0.03045
## alternative hypothesis: true mean is not equal to 24
## 95 percent confidence interval:
##  17.37768 23.65089
## sample estimates:
## mean of x 
##  20.51429
\end{verbatim}

\begin{Shaded}
\begin{Highlighting}[]
\CommentTok{\#3.H0 {-} 50 процента са доволни}
\CommentTok{\# H1 {-} \textless{} 50 процента са доволни}

\FunctionTok{prop.test}\NormalTok{(}\DecValTok{42}\NormalTok{, }\DecValTok{100}\NormalTok{, }\AttributeTok{p =} \FloatTok{0.5}\NormalTok{, }\AttributeTok{alternative =} \StringTok{\textquotesingle{}less\textquotesingle{}}\NormalTok{)}
\end{Highlighting}
\end{Shaded}

\begin{verbatim}
## 
##  1-sample proportions test with continuity correction
## 
## data:  42 out of 100, null probability 0.5
## X-squared = 2.25, df = 1, p-value = 0.06681
## alternative hypothesis: true p is less than 0.5
## 95 percent confidence interval:
##  0.0000000 0.5072341
## sample estimates:
##    p 
## 0.42
\end{verbatim}

\begin{Shaded}
\begin{Highlighting}[]
\CommentTok{\#4.H0 {-} 5 minutes on the phone H1 {-} more than 5 mins}
\NormalTok{data4 }\OtherTok{\textless{}{-}} \FunctionTok{c}\NormalTok{(}\FloatTok{12.8}\NormalTok{, }\FloatTok{3.5}\NormalTok{, }\FloatTok{2.9}\NormalTok{, }\FloatTok{9.4}\NormalTok{, }\FloatTok{8.7}\NormalTok{, }\FloatTok{0.7}\NormalTok{, }\FloatTok{0.2}\NormalTok{, }\FloatTok{2.8}\NormalTok{, }\FloatTok{1.9}\NormalTok{, }\FloatTok{2.8}\NormalTok{, }\FloatTok{3.1}\NormalTok{, }\FloatTok{15.8}\NormalTok{)}

\FunctionTok{qqnorm}\NormalTok{(data4)}
\FunctionTok{qqline}\NormalTok{(data4)}
\end{Highlighting}
\end{Shaded}

\includegraphics{examPrep_files/figure-latex/unnamed-chunk-32-1.pdf}

\begin{Shaded}
\begin{Highlighting}[]
\CommentTok{\#not normal distribution}
\FunctionTok{shapiro.test}\NormalTok{(data4)}
\end{Highlighting}
\end{Shaded}

\begin{verbatim}
## 
##  Shapiro-Wilk normality test
## 
## data:  data4
## W = 0.83988, p-value = 0.0276
\end{verbatim}

\begin{Shaded}
\begin{Highlighting}[]
\CommentTok{\#we accept h0}
\FunctionTok{wilcox.test}\NormalTok{(data4, }\AttributeTok{mu =} \DecValTok{5}\NormalTok{, }\AttributeTok{alternative =} \StringTok{\textquotesingle{}greater\textquotesingle{}}\NormalTok{)}
\end{Highlighting}
\end{Shaded}

\begin{verbatim}
## Warning in wilcox.test.default(data4, mu = 5, alternative = "greater"): cannot
## compute exact p-value with ties
\end{verbatim}

\begin{verbatim}
## 
##  Wilcoxon signed rank test with continuity correction
## 
## data:  data4
## V = 39, p-value = 0.5156
## alternative hypothesis: true location is greater than 5
\end{verbatim}

\begin{Shaded}
\begin{Highlighting}[]
\CommentTok{\#5.h0 {-} live more than 100 days h1{-} less than 100 days}
\NormalTok{data5 }\OtherTok{\textless{}{-}}\NormalTok{ cancer}\SpecialCharTok{$}\NormalTok{stomach}

\FunctionTok{qqnorm}\NormalTok{(data5)}
\FunctionTok{qqline}\NormalTok{(data5)}
\end{Highlighting}
\end{Shaded}

\includegraphics{examPrep_files/figure-latex/unnamed-chunk-33-1.pdf}

\begin{Shaded}
\begin{Highlighting}[]
\CommentTok{\#not normal distr}
\FunctionTok{shapiro.test}\NormalTok{(data5)}
\end{Highlighting}
\end{Shaded}

\begin{verbatim}
## 
##  Shapiro-Wilk normality test
## 
## data:  data5
## W = 0.75473, p-value = 0.002075
\end{verbatim}

\begin{Shaded}
\begin{Highlighting}[]
\CommentTok{\#we accept h0}
\FunctionTok{wilcox.test}\NormalTok{(data5, }\AttributeTok{mu =} \DecValTok{100}\NormalTok{, }\AttributeTok{alternative =} \StringTok{\textquotesingle{}less\textquotesingle{}}\NormalTok{)}
\end{Highlighting}
\end{Shaded}

\begin{verbatim}
## 
##  Wilcoxon signed rank exact test
## 
## data:  data5
## V = 61, p-value = 0.8633
## alternative hypothesis: true location is less than 100
\end{verbatim}

От упражение 9

\begin{Shaded}
\begin{Highlighting}[]
\CommentTok{\#1. H0 {-} equal h1 {-} not equal}
\NormalTok{x }\OtherTok{\textless{}{-}} \FunctionTok{c}\NormalTok{(}\DecValTok{4}\NormalTok{, }\DecValTok{1}\NormalTok{, }\DecValTok{7}\NormalTok{, }\DecValTok{9}\NormalTok{)}
\NormalTok{y }\OtherTok{\textless{}{-}} \FunctionTok{c}\NormalTok{(}\DecValTok{10}\NormalTok{, }\DecValTok{3}\NormalTok{, }\DecValTok{2}\NormalTok{, }\DecValTok{11}\NormalTok{)}

\CommentTok{\#не са норм разпр}
\FunctionTok{qqnorm}\NormalTok{(x)}
\FunctionTok{qqline}\NormalTok{(x)}
\end{Highlighting}
\end{Shaded}

\includegraphics{examPrep_files/figure-latex/unnamed-chunk-34-1.pdf}

\begin{Shaded}
\begin{Highlighting}[]
\FunctionTok{qqnorm}\NormalTok{(y)}
\FunctionTok{qqline}\NormalTok{(y)}
\end{Highlighting}
\end{Shaded}

\includegraphics{examPrep_files/figure-latex/unnamed-chunk-34-2.pdf}

\begin{Shaded}
\begin{Highlighting}[]
\CommentTok{\#приемаме нулевата хипотеза}
\FunctionTok{wilcox.test}\NormalTok{(x, y, }\AttributeTok{alternative =} \StringTok{\textquotesingle{}two.sided\textquotesingle{}}\NormalTok{)}
\end{Highlighting}
\end{Shaded}

\begin{verbatim}
## 
##  Wilcoxon rank sum exact test
## 
## data:  x and y
## W = 6, p-value = 0.6857
## alternative hypothesis: true location shift is not equal to 0
\end{verbatim}

\begin{Shaded}
\begin{Highlighting}[]
\CommentTok{\#2.H0 {-} equal H1 {-} greater}

\CommentTok{\#we accept h1}
\FunctionTok{prop.test}\NormalTok{(}\FunctionTok{c}\NormalTok{(}\DecValTok{351}\NormalTok{, }\DecValTok{71}\NormalTok{), }\FunctionTok{c}\NormalTok{(}\DecValTok{605}\NormalTok{, }\DecValTok{195}\NormalTok{), }\AttributeTok{alternative =} \StringTok{\textquotesingle{}greater\textquotesingle{}}\NormalTok{)}
\end{Highlighting}
\end{Shaded}

\begin{verbatim}
## 
##  2-sample test for equality of proportions with continuity correction
## 
## data:  c(351, 71) out of c(605, 195)
## X-squared = 26.761, df = 1, p-value = 1.151e-07
## alternative hypothesis: greater
## 95 percent confidence interval:
##  0.1470851 1.0000000
## sample estimates:
##    prop 1    prop 2 
## 0.5801653 0.3641026
\end{verbatim}

\begin{Shaded}
\begin{Highlighting}[]
\CommentTok{\#3.H0 {-} less H1 {-} more}

\NormalTok{before }\OtherTok{\textless{}{-}} \FunctionTok{c}\NormalTok{( }\DecValTok{15}\NormalTok{, }\DecValTok{10}\NormalTok{, }\DecValTok{13}\NormalTok{, }\DecValTok{7}\NormalTok{, }\DecValTok{9}\NormalTok{, }\DecValTok{8}\NormalTok{, }\DecValTok{21}\NormalTok{, }\DecValTok{9}\NormalTok{, }\DecValTok{14}\NormalTok{, }\DecValTok{8}\NormalTok{)}
\NormalTok{after }\OtherTok{\textless{}{-}} \FunctionTok{c}\NormalTok{(}\DecValTok{15}\NormalTok{, }\DecValTok{14}\NormalTok{, }\DecValTok{12}\NormalTok{, }\DecValTok{8}\NormalTok{, }\DecValTok{14}\NormalTok{, }\DecValTok{10}\NormalTok{, }\DecValTok{7}\NormalTok{, }\DecValTok{16}\NormalTok{, }\DecValTok{10}\NormalTok{, }\DecValTok{15}\NormalTok{, }\DecValTok{12}\NormalTok{)}

\CommentTok{\#normal distr}
\FunctionTok{qqnorm}\NormalTok{(before)}
\FunctionTok{qqline}\NormalTok{(before)}
\end{Highlighting}
\end{Shaded}

\includegraphics{examPrep_files/figure-latex/unnamed-chunk-36-1.pdf}

\begin{Shaded}
\begin{Highlighting}[]
\FunctionTok{qqnorm}\NormalTok{(after)}
\FunctionTok{qqline}\NormalTok{(after)}
\end{Highlighting}
\end{Shaded}

\includegraphics{examPrep_files/figure-latex/unnamed-chunk-36-2.pdf}

\begin{Shaded}
\begin{Highlighting}[]
\FunctionTok{t.test}\NormalTok{(before, after, }\AttributeTok{alternative =} \StringTok{\textquotesingle{}greater\textquotesingle{}}\NormalTok{)}
\end{Highlighting}
\end{Shaded}

\begin{verbatim}
## 
##  Welch Two Sample t-test
## 
## data:  before and after
## t = -0.41894, df = 15.853, p-value = 0.6596
## alternative hypothesis: true difference in means is greater than 0
## 95 percent confidence interval:
##  -3.571812       Inf
## sample estimates:
## mean of x mean of y 
##  11.40000  12.09091
\end{verbatim}

\begin{Shaded}
\begin{Highlighting}[]
\CommentTok{\#4.Х0 {-} че са равни Х1 {-} че са различни }

\NormalTok{radar1 }\OtherTok{\textless{}{-}} \FunctionTok{c}\NormalTok{(}\DecValTok{70}\NormalTok{, }\DecValTok{85}\NormalTok{, }\DecValTok{63}\NormalTok{, }\DecValTok{54}\NormalTok{, }\DecValTok{65}\NormalTok{, }\DecValTok{80}\NormalTok{, }\DecValTok{75}\NormalTok{, }\DecValTok{95}\NormalTok{, }\DecValTok{52}\NormalTok{, }\DecValTok{55}\NormalTok{)}

\NormalTok{radar2 }\OtherTok{\textless{}{-}} \FunctionTok{c}\NormalTok{(}\DecValTok{72}\NormalTok{, }\DecValTok{86}\NormalTok{, }\DecValTok{62}\NormalTok{, }\DecValTok{55}\NormalTok{, }\DecValTok{63}\NormalTok{, }\DecValTok{80}\NormalTok{, }\DecValTok{78}\NormalTok{, }\DecValTok{90}\NormalTok{, }\DecValTok{53}\NormalTok{, }\DecValTok{57}\NormalTok{)}

\CommentTok{\#they are normally distributed}
\FunctionTok{qqnorm}\NormalTok{(radar1)}
\FunctionTok{qqline}\NormalTok{(radar1)}
\end{Highlighting}
\end{Shaded}

\includegraphics{examPrep_files/figure-latex/unnamed-chunk-37-1.pdf}

\begin{Shaded}
\begin{Highlighting}[]
\FunctionTok{qqnorm}\NormalTok{(radar2)}
\FunctionTok{qqline}\NormalTok{(radar2)}
\end{Highlighting}
\end{Shaded}

\includegraphics{examPrep_files/figure-latex/unnamed-chunk-37-2.pdf}

\begin{Shaded}
\begin{Highlighting}[]
\FunctionTok{t.test}\NormalTok{(radar1, radar2, }\AttributeTok{alternative =} \StringTok{\textquotesingle{}two.sided\textquotesingle{}}\NormalTok{, }\AttributeTok{paired =}\NormalTok{ T)}
\end{Highlighting}
\end{Shaded}

\begin{verbatim}
## 
##  Paired t-test
## 
## data:  radar1 and radar2
## t = -0.26941, df = 9, p-value = 0.7937
## alternative hypothesis: true difference in means is not equal to 0
## 95 percent confidence interval:
##  -1.879354  1.479354
## sample estimates:
## mean of the differences 
##                    -0.2
\end{verbatim}

От упражнение 10

\begin{Shaded}
\begin{Highlighting}[]
\CommentTok{\#1.H0 {-} data elements are equal H1{-} they are not}

\NormalTok{data6 }\OtherTok{\textless{}{-}} \FunctionTok{c}\NormalTok{(}\DecValTok{125}\NormalTok{, }\DecValTok{410}\NormalTok{, }\DecValTok{310}\NormalTok{, }\DecValTok{300}\NormalTok{, }\DecValTok{318}\NormalTok{, }\DecValTok{298}\NormalTok{, }\DecValTok{148}\NormalTok{)}

\CommentTok{\#we cant accept H0}
\FunctionTok{chisq.test}\NormalTok{(data6)}
\end{Highlighting}
\end{Shaded}

\begin{verbatim}
## 
##  Chi-squared test for given probabilities
## 
## data:  data6
## X-squared = 223.84, df = 6, p-value < 2.2e-16
\end{verbatim}

\begin{Shaded}
\begin{Highlighting}[]
\CommentTok{\#2.H0 {-} all digits have the same prob H1 {-} they don\textquotesingle{}t  }

\CommentTok{\#взимаме първите 200 цифри и виждаме колко често всяка една се среща}
\NormalTok{data7 }\OtherTok{\textless{}{-}}\NormalTok{ pi2000[}\DecValTok{1}\SpecialCharTok{:}\DecValTok{200}\NormalTok{] }\SpecialCharTok{\%\textgreater{}\%} \FunctionTok{table}\NormalTok{()}

\CommentTok{\#правим хи{-}квадрат теста}
\CommentTok{\#можем да приемем нулевата хипотеза}
\FunctionTok{chisq.test}\NormalTok{(data7)}
\end{Highlighting}
\end{Shaded}

\begin{verbatim}
## 
##  Chi-squared test for given probabilities
## 
## data:  data7
## X-squared = 7.2, df = 9, p-value = 0.6163
\end{verbatim}

\begin{Shaded}
\begin{Highlighting}[]
\CommentTok{\#3.}
\CommentTok{\#теоритичните вероятности за срещането на буквите в англ език}
\NormalTok{probs }\OtherTok{\textless{}{-}} \FunctionTok{c}\NormalTok{(}\FloatTok{0.1270}\NormalTok{, }\FloatTok{0.0956}\NormalTok{, }\FloatTok{0.0817}\NormalTok{, }\FloatTok{0.0751}\NormalTok{, }\FloatTok{0.0697}\NormalTok{, }\FloatTok{0.0675}\NormalTok{, }\FloatTok{0.4834}\NormalTok{)}

\CommentTok{\#срещането на буквите от нашия текст}
\NormalTok{letters }\OtherTok{\textless{}{-}} \FunctionTok{c}\NormalTok{(}\DecValTok{102}\NormalTok{, }\DecValTok{108}\NormalTok{, }\DecValTok{90}\NormalTok{, }\DecValTok{95}\NormalTok{, }\DecValTok{82}\NormalTok{, }\DecValTok{40}\NormalTok{, }\DecValTok{519}\NormalTok{)}

\CommentTok{\#правим проверка дали са равни}
\FunctionTok{chisq.test}\NormalTok{(letters, }\AttributeTok{p =}\NormalTok{ probs)}
\end{Highlighting}
\end{Shaded}

\begin{verbatim}
## 
##  Chi-squared test for given probabilities
## 
## data:  letters
## X-squared = 26.396, df = 6, p-value = 0.0001878
\end{verbatim}

\begin{Shaded}
\begin{Highlighting}[]
\CommentTok{\#p{-}value е много малко затова отхвърляме хипотезата}
\end{Highlighting}
\end{Shaded}

\begin{Shaded}
\begin{Highlighting}[]
\CommentTok{\#4.искаме да видим дали колан / без колан са независими сл.в Това става като подадед}
\CommentTok{\#на ф{-}ята матрица и тя прави проверката, т.е нулевата хипотеза е, че данните са независими}
\CommentTok{\# иначе са зависими}

\NormalTok{belt }\OtherTok{\textless{}{-}} \FunctionTok{c}\NormalTok{(}\DecValTok{12813}\NormalTok{, }\DecValTok{647}\NormalTok{, }\DecValTok{359}\NormalTok{, }\DecValTok{42}\NormalTok{)}

\NormalTok{nobelt }\OtherTok{\textless{}{-}} \FunctionTok{c}\NormalTok{(}\DecValTok{65963}\NormalTok{, }\DecValTok{4000}\NormalTok{, }\DecValTok{2642}\NormalTok{, }\DecValTok{303}\NormalTok{)}

\NormalTok{data8 }\OtherTok{\textless{}{-}} \FunctionTok{matrix}\NormalTok{(belt, }\AttributeTok{nrow =} \DecValTok{1}\NormalTok{, }\AttributeTok{ncol =} \DecValTok{4}\NormalTok{)}

\NormalTok{data9 }\OtherTok{\textless{}{-}} \FunctionTok{rbind}\NormalTok{(data8, nobelt)}

\CommentTok{\#p{-}value е много малко следователно отхвърляме нулевата хипотеза, т.е са зависими}
\CommentTok{\#колан влияе на нараняването при катастрофа}
\FunctionTok{chisq.test}\NormalTok{(data9)}
\end{Highlighting}
\end{Shaded}

\begin{verbatim}
## 
##  Pearson's Chi-squared test
## 
## data:  data9
## X-squared = 59.224, df = 3, p-value = 8.61e-13
\end{verbatim}

\begin{Shaded}
\begin{Highlighting}[]
\CommentTok{\#5.}

\NormalTok{mon }\OtherTok{\textless{}{-}} \FunctionTok{c}\NormalTok{(}\DecValTok{44}\NormalTok{, }\DecValTok{14}\NormalTok{, }\DecValTok{15}\NormalTok{, }\DecValTok{3}\NormalTok{)}
\NormalTok{tues }\OtherTok{\textless{}{-}} \FunctionTok{c}\NormalTok{(}\DecValTok{74}\NormalTok{, }\DecValTok{25}\NormalTok{, }\DecValTok{20}\NormalTok{, }\DecValTok{5}\NormalTok{)}
\NormalTok{wed }\OtherTok{\textless{}{-}} \FunctionTok{c}\NormalTok{(}\DecValTok{79}\NormalTok{, }\DecValTok{27}\NormalTok{, }\DecValTok{20}\NormalTok{, }\DecValTok{5}\NormalTok{)}
\NormalTok{thurs }\OtherTok{\textless{}{-}} \FunctionTok{c}\NormalTok{(}\DecValTok{72}\NormalTok{, }\DecValTok{24}\NormalTok{, }\DecValTok{23}\NormalTok{, }\DecValTok{0}\NormalTok{)}
\NormalTok{fri }\OtherTok{\textless{}{-}} \FunctionTok{c}\NormalTok{(}\DecValTok{31}\NormalTok{, }\DecValTok{10}\NormalTok{, }\DecValTok{9}\NormalTok{, }\DecValTok{0}\NormalTok{)}

\NormalTok{final }\OtherTok{\textless{}{-}} \FunctionTok{matrix}\NormalTok{(}\FunctionTok{c}\NormalTok{(mon, tues, wed, thurs, fri), }\AttributeTok{nrow =} \DecValTok{5}\NormalTok{)}

\CommentTok{\#отхвърляме нулевата хипотеза че са независими , т.е има връзка между деня и качеството на стоката}
\FunctionTok{chisq.test}\NormalTok{(final)}
\end{Highlighting}
\end{Shaded}

\begin{verbatim}
## 
##  Pearson's Chi-squared test
## 
## data:  final
## X-squared = 350.71, df = 12, p-value < 2.2e-16
\end{verbatim}

От упражнение 11

\begin{Shaded}
\begin{Highlighting}[]
\CommentTok{\#1.Създаваме дата фрейм за данните}
\NormalTok{patients\_df }\OtherTok{\textless{}{-}} \FunctionTok{data.frame}\NormalTok{(}
  \AttributeTok{age =} \FunctionTok{c}\NormalTok{(}\DecValTok{18}\NormalTok{, }\DecValTok{23}\NormalTok{, }\DecValTok{25}\NormalTok{, }\DecValTok{35}\NormalTok{, }\DecValTok{65}\NormalTok{, }\DecValTok{54}\NormalTok{, }\DecValTok{34}\NormalTok{, }\DecValTok{56}\NormalTok{, }\DecValTok{72}\NormalTok{, }\DecValTok{19}\NormalTok{, }\DecValTok{23}\NormalTok{, }\DecValTok{42}\NormalTok{, }\DecValTok{18}\NormalTok{, }\DecValTok{39}\NormalTok{, }\DecValTok{37}\NormalTok{),}
  \AttributeTok{max\_pulse =} \FunctionTok{c}\NormalTok{(}\DecValTok{202}\NormalTok{, }\DecValTok{186}\NormalTok{, }\DecValTok{187}\NormalTok{, }\DecValTok{180}\NormalTok{, }\DecValTok{156}\NormalTok{, }\DecValTok{169}\NormalTok{, }\DecValTok{174}\NormalTok{, }\DecValTok{172}\NormalTok{, }\DecValTok{153}\NormalTok{, }\DecValTok{199}\NormalTok{, }\DecValTok{193}\NormalTok{, }\DecValTok{174}\NormalTok{, }\DecValTok{198}\NormalTok{, }\DecValTok{183}\NormalTok{, }\DecValTok{178}\NormalTok{)}
\NormalTok{)}

\CommentTok{\#това е за модел на линейната регресия}
\NormalTok{model1 }\OtherTok{\textless{}{-}} \FunctionTok{lm}\NormalTok{(patients\_df}\SpecialCharTok{$}\NormalTok{max\_pulse }\SpecialCharTok{\textasciitilde{}}\NormalTok{ patients\_df}\SpecialCharTok{$}\NormalTok{age, }\AttributeTok{data =}\NormalTok{ patients\_df)}


\FunctionTok{plot}\NormalTok{(patients\_df)}
\FunctionTok{abline}\NormalTok{(model1, }\AttributeTok{col =} \StringTok{"red"}\NormalTok{)}
\end{Highlighting}
\end{Shaded}

\includegraphics{examPrep_files/figure-latex/unnamed-chunk-43-1.pdf}

\begin{Shaded}
\begin{Highlighting}[]
\NormalTok{summ\_lm }\OtherTok{\textless{}{-}} \FunctionTok{summary}\NormalTok{(model1)}

\NormalTok{n }\OtherTok{\textless{}{-}} \FunctionTok{nrow}\NormalTok{(patients\_df)}

\CommentTok{\# Тестване на хипотезата, че бета1 = {-}1}
\CommentTok{\# H0 :{-} "бета\_1 = {-}1"}

\CommentTok{\# Стандартно отклонение(грешка) на оценката за бета1}
\NormalTok{std\_b1 }\OtherTok{\textless{}{-}}\NormalTok{ summ\_lm}\SpecialCharTok{$}\NormalTok{coefficients[}\DecValTok{2}\NormalTok{, }\DecValTok{2}\NormalTok{]}

\CommentTok{\# оценката за бета1}
\NormalTok{est\_b1 }\OtherTok{\textless{}{-}}\NormalTok{ summ\_lm}\SpecialCharTok{$}\NormalTok{coefficients[}\DecValTok{2}\NormalTok{, }\DecValTok{1}\NormalTok{]}


\CommentTok{\# Параметър за бета1 под нулева хипотеза}
\NormalTok{b1\_null\_hyp }\OtherTok{\textless{}{-}} \SpecialCharTok{{-}}\DecValTok{1}

\CommentTok{\# Изграждане на т{-}статистика  }
\NormalTok{t\_statistic }\OtherTok{\textless{}{-}}\NormalTok{ (est\_b1 }\SpecialCharTok{{-}}\NormalTok{ b1\_null\_hyp) }\SpecialCharTok{/}\NormalTok{ std\_b1}

\CommentTok{\# Вероятност да наблюдваме тази т{-}статистика (или по{-}крайна) при положение, че е вярна нулевата хипотеза}
\NormalTok{pval }\OtherTok{\textless{}{-}} \DecValTok{2} \SpecialCharTok{*} \FunctionTok{pt}\NormalTok{(t\_statistic, n }\SpecialCharTok{{-}} \DecValTok{2}\NormalTok{, }\AttributeTok{lower.tail =} \ConstantTok{FALSE}\NormalTok{)}


\CommentTok{\# Прогнозиране за възрасти 30, 40 и 50}
\FunctionTok{predict.lm}\NormalTok{(}
\NormalTok{  model1,}
  \AttributeTok{newdata =} \FunctionTok{data.frame}\NormalTok{(}\AttributeTok{age =} \FunctionTok{c}\NormalTok{(}\DecValTok{30}\NormalTok{, }\DecValTok{40}\NormalTok{, }\DecValTok{50}\NormalTok{)),}
  \AttributeTok{interval =} \StringTok{"confidence"}\NormalTok{,}
  \AttributeTok{level =} \FloatTok{0.9}
\NormalTok{)}
\end{Highlighting}
\end{Shaded}

\begin{verbatim}
## Warning: 'newdata' had 3 rows but variables found have 15 rows
\end{verbatim}

\begin{verbatim}
##         fit      lwr      upr
## 1  195.6894 192.5083 198.8705
## 2  191.7007 188.9557 194.4458
## 3  190.1053 187.5137 192.6969
## 4  182.1280 180.0149 184.2411
## 5  158.1962 154.1798 162.2127
## 6  166.9712 164.0309 169.9116
## 7  182.9258 180.7922 185.0593
## 8  165.3758 162.2564 168.4952
## 9  152.6121 147.8341 157.3902
## 10 194.8917 191.8028 197.9805
## 11 191.7007 188.9557 194.4458
## 12 176.5439 174.3723 178.7155
## 13 195.6894 192.5083 198.8705
## 14 178.9371 176.8337 181.0405
## 15 180.5326 178.4390 182.6262
\end{verbatim}

Oт упражнение 12

От упражнение 13

\begin{Shaded}
\begin{Highlighting}[]
\CommentTok{\#нулевата хипотеза е че имат равни средни трите извадки от данни}
\CommentTok{\#правим дата фрейм с данните от задачата}
\NormalTok{  exams\_df }\OtherTok{\textless{}{-}} \FunctionTok{data.frame}\NormalTok{(}
\AttributeTok{examinor1 =} \FunctionTok{c}\NormalTok{(}\DecValTok{5}\NormalTok{, }\DecValTok{4}\NormalTok{, }\DecValTok{4}\NormalTok{, }\DecValTok{6}\NormalTok{, }\DecValTok{4}\NormalTok{, }\DecValTok{6}\NormalTok{, }\DecValTok{3}\NormalTok{, }\DecValTok{3}\NormalTok{, }\DecValTok{4}\NormalTok{, }\DecValTok{5}\NormalTok{),}
\AttributeTok{examinor2 =} \FunctionTok{c}\NormalTok{(}\DecValTok{3}\NormalTok{, }\DecValTok{2}\NormalTok{, }\DecValTok{4}\NormalTok{, }\DecValTok{5}\NormalTok{, }\DecValTok{3}\NormalTok{, }\DecValTok{4}\NormalTok{, }\DecValTok{3}\NormalTok{, }\DecValTok{4}\NormalTok{, }\DecValTok{2}\NormalTok{, }\DecValTok{4}\NormalTok{),}
\AttributeTok{examinor3 =} \FunctionTok{c}\NormalTok{(}\DecValTok{4}\NormalTok{ ,}\DecValTok{6}\NormalTok{ ,}\DecValTok{4}\NormalTok{ ,}\DecValTok{2}\NormalTok{ ,}\DecValTok{4}\NormalTok{ ,}\DecValTok{5}\NormalTok{ ,}\DecValTok{5}\NormalTok{ ,}\DecValTok{3}\NormalTok{ ,}\DecValTok{6}\NormalTok{ ,}\DecValTok{4}\NormalTok{)}
\NormalTok{) }

\NormalTok{ stacked\_exam\_df }\OtherTok{\textless{}{-}} \FunctionTok{stack}\NormalTok{(exams\_df)}
 
\CommentTok{\#гледаме дали са нормално разпределени данните}
 
\FunctionTok{qqnorm}\NormalTok{(exams\_df}\SpecialCharTok{$}\NormalTok{examinor1)}
\FunctionTok{qqline}\NormalTok{(exams\_df}\SpecialCharTok{$}\NormalTok{examinor1)}
\end{Highlighting}
\end{Shaded}

\includegraphics{examPrep_files/figure-latex/unnamed-chunk-45-1.pdf}

\begin{Shaded}
\begin{Highlighting}[]
\FunctionTok{shapiro.test}\NormalTok{(exams\_df}\SpecialCharTok{$}\NormalTok{examinor1)}
\end{Highlighting}
\end{Shaded}

\begin{verbatim}
## 
##  Shapiro-Wilk normality test
## 
## data:  exams_df$examinor1
## W = 0.89165, p-value = 0.177
\end{verbatim}

\begin{Shaded}
\begin{Highlighting}[]
\FunctionTok{qqnorm}\NormalTok{(exams\_df}\SpecialCharTok{$}\NormalTok{examinor2)}
\FunctionTok{qqline}\NormalTok{(exams\_df}\SpecialCharTok{$}\NormalTok{examinor2)}
\end{Highlighting}
\end{Shaded}

\includegraphics{examPrep_files/figure-latex/unnamed-chunk-45-2.pdf}

\begin{Shaded}
\begin{Highlighting}[]
\FunctionTok{shapiro.test}\NormalTok{(exams\_df}\SpecialCharTok{$}\NormalTok{examinor2)}
\end{Highlighting}
\end{Shaded}

\begin{verbatim}
## 
##  Shapiro-Wilk normality test
## 
## data:  exams_df$examinor2
## W = 0.90444, p-value = 0.2449
\end{verbatim}

\begin{Shaded}
\begin{Highlighting}[]
\FunctionTok{qqnorm}\NormalTok{(exams\_df}\SpecialCharTok{$}\NormalTok{examinor3)}
\FunctionTok{qqline}\NormalTok{(exams\_df}\SpecialCharTok{$}\NormalTok{examinor3)}
\end{Highlighting}
\end{Shaded}

\includegraphics{examPrep_files/figure-latex/unnamed-chunk-45-3.pdf}

\begin{Shaded}
\begin{Highlighting}[]
\FunctionTok{shapiro.test}\NormalTok{(exams\_df}\SpecialCharTok{$}\NormalTok{examinor3)}
\end{Highlighting}
\end{Shaded}

\begin{verbatim}
## 
##  Shapiro-Wilk normality test
## 
## data:  exams_df$examinor3
## W = 0.92883, p-value = 0.4365
\end{verbatim}

\begin{Shaded}
\begin{Highlighting}[]
\CommentTok{\#и трите са нормално  разпределени тоест можем да направим тест дали имат еднакво средно ако са норм разпр}

\FunctionTok{oneway.test}\NormalTok{(values }\SpecialCharTok{\textasciitilde{}}\NormalTok{ ind, }\AttributeTok{data =}\NormalTok{ stacked\_exam\_df)}
\end{Highlighting}
\end{Shaded}

\begin{verbatim}
## 
##  One-way analysis of means (not assuming equal variances)
## 
## data:  values and ind
## F = 2.7825, num df = 2.000, denom df = 17.811, p-value = 0.0888
\end{verbatim}

\begin{Shaded}
\begin{Highlighting}[]
\CommentTok{\#Не можем да отхвърлим хипотезата че имат равни средни}
\CommentTok{\#друг начин да се провери съшата хипотеза}

\FunctionTok{anova}\NormalTok{(}\FunctionTok{lm}\NormalTok{(values }\SpecialCharTok{\textasciitilde{}}\NormalTok{ ind, }\AttributeTok{data =}\NormalTok{ stacked\_exam\_df))}
\end{Highlighting}
\end{Shaded}

\begin{verbatim}
## Analysis of Variance Table
## 
## Response: values
##           Df Sum Sq Mean Sq F value Pr(>F)
## ind        2  6.067  3.0333  2.4894 0.1018
## Residuals 27 32.900  1.2185
\end{verbatim}

\begin{Shaded}
\begin{Highlighting}[]
\CommentTok{\#2.}

\NormalTok{groupC }\OtherTok{\textless{}{-}}\NormalTok{ InsectSprays}\SpecialCharTok{$}\NormalTok{count[InsectSprays}\SpecialCharTok{$}\NormalTok{spray }\SpecialCharTok{==} \StringTok{\textquotesingle{}C\textquotesingle{}}\NormalTok{]}

\NormalTok{groupD }\OtherTok{\textless{}{-}}\NormalTok{ InsectSprays}\SpecialCharTok{$}\NormalTok{count[InsectSprays}\SpecialCharTok{$}\NormalTok{spray }\SpecialCharTok{==} \StringTok{\textquotesingle{}D\textquotesingle{}}\NormalTok{]}

\NormalTok{groupE }\OtherTok{\textless{}{-}}\NormalTok{ InsectSprays}\SpecialCharTok{$}\NormalTok{count[InsectSprays}\SpecialCharTok{$}\NormalTok{spray }\SpecialCharTok{==} \StringTok{\textquotesingle{}E\textquotesingle{}}\NormalTok{]}

\CommentTok{\#изглежда ми сравнително нормално разпределени}
\FunctionTok{qqnorm}\NormalTok{(groupC)}
\FunctionTok{qqline}\NormalTok{(groupC)}
\end{Highlighting}
\end{Shaded}

\includegraphics{examPrep_files/figure-latex/unnamed-chunk-46-1.pdf}

\begin{Shaded}
\begin{Highlighting}[]
\FunctionTok{shapiro.test}\NormalTok{(groupC)}
\end{Highlighting}
\end{Shaded}

\begin{verbatim}
## 
##  Shapiro-Wilk normality test
## 
## data:  groupC
## W = 0.85907, p-value = 0.04759
\end{verbatim}

\begin{Shaded}
\begin{Highlighting}[]
\FunctionTok{qqnorm}\NormalTok{(groupD)}
\FunctionTok{qqline}\NormalTok{(groupD)}
\end{Highlighting}
\end{Shaded}

\includegraphics{examPrep_files/figure-latex/unnamed-chunk-46-2.pdf}

\begin{Shaded}
\begin{Highlighting}[]
\FunctionTok{shapiro.test}\NormalTok{(groupD)}
\end{Highlighting}
\end{Shaded}

\begin{verbatim}
## 
##  Shapiro-Wilk normality test
## 
## data:  groupD
## W = 0.75063, p-value = 0.002713
\end{verbatim}

\begin{Shaded}
\begin{Highlighting}[]
\FunctionTok{qqnorm}\NormalTok{(groupE)}
\FunctionTok{qqline}\NormalTok{(groupE)}
\end{Highlighting}
\end{Shaded}

\includegraphics{examPrep_files/figure-latex/unnamed-chunk-46-3.pdf}

\begin{Shaded}
\begin{Highlighting}[]
\FunctionTok{shapiro.test}\NormalTok{(groupE)}
\end{Highlighting}
\end{Shaded}

\begin{verbatim}
## 
##  Shapiro-Wilk normality test
## 
## data:  groupE
## W = 0.92128, p-value = 0.2967
\end{verbatim}

\begin{Shaded}
\begin{Highlighting}[]
\CommentTok{\#p{-}value{-}то е много малко следователно можем да твърдим че някои от препаратите действат по{-}добре от други}
\FunctionTok{oneway.test}\NormalTok{(count }\SpecialCharTok{\textasciitilde{}}\NormalTok{ spray, }\AttributeTok{data =}\NormalTok{ InsectSprays)}
\end{Highlighting}
\end{Shaded}

\begin{verbatim}
## 
##  One-way analysis of means (not assuming equal variances)
## 
## data:  count and spray
## F = 36.065, num df = 5.000, denom df = 30.043, p-value = 7.999e-12
\end{verbatim}

\begin{Shaded}
\begin{Highlighting}[]
\CommentTok{\#3.взимаме данните от файла}
\NormalTok{drug\_df }\OtherTok{\textless{}{-}} \FunctionTok{read.csv}\NormalTok{(}\StringTok{"./data.txt"}\NormalTok{)}

\CommentTok{\#тъй като имаме сдвоени данни, т.е даваме лекартво на един и същ пациент ползваме aov}
\FunctionTok{aov}\NormalTok{(response }\SpecialCharTok{\textasciitilde{}}\NormalTok{ drug }\SpecialCharTok{+} \FunctionTok{Error}\NormalTok{(patient), }\AttributeTok{data =}\NormalTok{ drug\_df) }\SpecialCharTok{\%\textgreater{}\%} \FunctionTok{summary}\NormalTok{()}
\end{Highlighting}
\end{Shaded}

\begin{verbatim}
## 
## Error: patient
##           Df Sum Sq Mean Sq F value Pr(>F)
## Residuals  1   72.9    72.9               
## 
## Error: Within
##           Df Sum Sq Mean Sq F value Pr(>F)
## drug       1     36   36.00   0.565  0.462
## Residuals 17   1082   63.66
\end{verbatim}

\begin{Shaded}
\begin{Highlighting}[]
\CommentTok{\#не можем да отхвърлим хопотезата, че имаме лекарствата действат еднакво}
\end{Highlighting}
\end{Shaded}

\begin{Shaded}
\begin{Highlighting}[]
\CommentTok{\#4.}
\NormalTok{iris}
\end{Highlighting}
\end{Shaded}

\begin{verbatim}
##     Sepal.Length Sepal.Width Petal.Length Petal.Width    Species
## 1            5.1         3.5          1.4         0.2     setosa
## 2            4.9         3.0          1.4         0.2     setosa
## 3            4.7         3.2          1.3         0.2     setosa
## 4            4.6         3.1          1.5         0.2     setosa
## 5            5.0         3.6          1.4         0.2     setosa
## 6            5.4         3.9          1.7         0.4     setosa
## 7            4.6         3.4          1.4         0.3     setosa
## 8            5.0         3.4          1.5         0.2     setosa
## 9            4.4         2.9          1.4         0.2     setosa
## 10           4.9         3.1          1.5         0.1     setosa
## 11           5.4         3.7          1.5         0.2     setosa
## 12           4.8         3.4          1.6         0.2     setosa
## 13           4.8         3.0          1.4         0.1     setosa
## 14           4.3         3.0          1.1         0.1     setosa
## 15           5.8         4.0          1.2         0.2     setosa
## 16           5.7         4.4          1.5         0.4     setosa
## 17           5.4         3.9          1.3         0.4     setosa
## 18           5.1         3.5          1.4         0.3     setosa
## 19           5.7         3.8          1.7         0.3     setosa
## 20           5.1         3.8          1.5         0.3     setosa
## 21           5.4         3.4          1.7         0.2     setosa
## 22           5.1         3.7          1.5         0.4     setosa
## 23           4.6         3.6          1.0         0.2     setosa
## 24           5.1         3.3          1.7         0.5     setosa
## 25           4.8         3.4          1.9         0.2     setosa
## 26           5.0         3.0          1.6         0.2     setosa
## 27           5.0         3.4          1.6         0.4     setosa
## 28           5.2         3.5          1.5         0.2     setosa
## 29           5.2         3.4          1.4         0.2     setosa
## 30           4.7         3.2          1.6         0.2     setosa
## 31           4.8         3.1          1.6         0.2     setosa
## 32           5.4         3.4          1.5         0.4     setosa
## 33           5.2         4.1          1.5         0.1     setosa
## 34           5.5         4.2          1.4         0.2     setosa
## 35           4.9         3.1          1.5         0.2     setosa
## 36           5.0         3.2          1.2         0.2     setosa
## 37           5.5         3.5          1.3         0.2     setosa
## 38           4.9         3.6          1.4         0.1     setosa
## 39           4.4         3.0          1.3         0.2     setosa
## 40           5.1         3.4          1.5         0.2     setosa
## 41           5.0         3.5          1.3         0.3     setosa
## 42           4.5         2.3          1.3         0.3     setosa
## 43           4.4         3.2          1.3         0.2     setosa
## 44           5.0         3.5          1.6         0.6     setosa
## 45           5.1         3.8          1.9         0.4     setosa
## 46           4.8         3.0          1.4         0.3     setosa
## 47           5.1         3.8          1.6         0.2     setosa
## 48           4.6         3.2          1.4         0.2     setosa
## 49           5.3         3.7          1.5         0.2     setosa
## 50           5.0         3.3          1.4         0.2     setosa
## 51           7.0         3.2          4.7         1.4 versicolor
## 52           6.4         3.2          4.5         1.5 versicolor
## 53           6.9         3.1          4.9         1.5 versicolor
## 54           5.5         2.3          4.0         1.3 versicolor
## 55           6.5         2.8          4.6         1.5 versicolor
## 56           5.7         2.8          4.5         1.3 versicolor
## 57           6.3         3.3          4.7         1.6 versicolor
## 58           4.9         2.4          3.3         1.0 versicolor
## 59           6.6         2.9          4.6         1.3 versicolor
## 60           5.2         2.7          3.9         1.4 versicolor
## 61           5.0         2.0          3.5         1.0 versicolor
## 62           5.9         3.0          4.2         1.5 versicolor
## 63           6.0         2.2          4.0         1.0 versicolor
## 64           6.1         2.9          4.7         1.4 versicolor
## 65           5.6         2.9          3.6         1.3 versicolor
## 66           6.7         3.1          4.4         1.4 versicolor
## 67           5.6         3.0          4.5         1.5 versicolor
## 68           5.8         2.7          4.1         1.0 versicolor
## 69           6.2         2.2          4.5         1.5 versicolor
## 70           5.6         2.5          3.9         1.1 versicolor
## 71           5.9         3.2          4.8         1.8 versicolor
## 72           6.1         2.8          4.0         1.3 versicolor
## 73           6.3         2.5          4.9         1.5 versicolor
## 74           6.1         2.8          4.7         1.2 versicolor
## 75           6.4         2.9          4.3         1.3 versicolor
## 76           6.6         3.0          4.4         1.4 versicolor
## 77           6.8         2.8          4.8         1.4 versicolor
## 78           6.7         3.0          5.0         1.7 versicolor
## 79           6.0         2.9          4.5         1.5 versicolor
## 80           5.7         2.6          3.5         1.0 versicolor
## 81           5.5         2.4          3.8         1.1 versicolor
## 82           5.5         2.4          3.7         1.0 versicolor
## 83           5.8         2.7          3.9         1.2 versicolor
## 84           6.0         2.7          5.1         1.6 versicolor
## 85           5.4         3.0          4.5         1.5 versicolor
## 86           6.0         3.4          4.5         1.6 versicolor
## 87           6.7         3.1          4.7         1.5 versicolor
## 88           6.3         2.3          4.4         1.3 versicolor
## 89           5.6         3.0          4.1         1.3 versicolor
## 90           5.5         2.5          4.0         1.3 versicolor
## 91           5.5         2.6          4.4         1.2 versicolor
## 92           6.1         3.0          4.6         1.4 versicolor
## 93           5.8         2.6          4.0         1.2 versicolor
## 94           5.0         2.3          3.3         1.0 versicolor
## 95           5.6         2.7          4.2         1.3 versicolor
## 96           5.7         3.0          4.2         1.2 versicolor
## 97           5.7         2.9          4.2         1.3 versicolor
## 98           6.2         2.9          4.3         1.3 versicolor
## 99           5.1         2.5          3.0         1.1 versicolor
## 100          5.7         2.8          4.1         1.3 versicolor
## 101          6.3         3.3          6.0         2.5  virginica
## 102          5.8         2.7          5.1         1.9  virginica
## 103          7.1         3.0          5.9         2.1  virginica
## 104          6.3         2.9          5.6         1.8  virginica
## 105          6.5         3.0          5.8         2.2  virginica
## 106          7.6         3.0          6.6         2.1  virginica
## 107          4.9         2.5          4.5         1.7  virginica
## 108          7.3         2.9          6.3         1.8  virginica
## 109          6.7         2.5          5.8         1.8  virginica
## 110          7.2         3.6          6.1         2.5  virginica
## 111          6.5         3.2          5.1         2.0  virginica
## 112          6.4         2.7          5.3         1.9  virginica
## 113          6.8         3.0          5.5         2.1  virginica
## 114          5.7         2.5          5.0         2.0  virginica
## 115          5.8         2.8          5.1         2.4  virginica
## 116          6.4         3.2          5.3         2.3  virginica
## 117          6.5         3.0          5.5         1.8  virginica
## 118          7.7         3.8          6.7         2.2  virginica
## 119          7.7         2.6          6.9         2.3  virginica
## 120          6.0         2.2          5.0         1.5  virginica
## 121          6.9         3.2          5.7         2.3  virginica
## 122          5.6         2.8          4.9         2.0  virginica
## 123          7.7         2.8          6.7         2.0  virginica
## 124          6.3         2.7          4.9         1.8  virginica
## 125          6.7         3.3          5.7         2.1  virginica
## 126          7.2         3.2          6.0         1.8  virginica
## 127          6.2         2.8          4.8         1.8  virginica
## 128          6.1         3.0          4.9         1.8  virginica
## 129          6.4         2.8          5.6         2.1  virginica
## 130          7.2         3.0          5.8         1.6  virginica
## 131          7.4         2.8          6.1         1.9  virginica
## 132          7.9         3.8          6.4         2.0  virginica
## 133          6.4         2.8          5.6         2.2  virginica
## 134          6.3         2.8          5.1         1.5  virginica
## 135          6.1         2.6          5.6         1.4  virginica
## 136          7.7         3.0          6.1         2.3  virginica
## 137          6.3         3.4          5.6         2.4  virginica
## 138          6.4         3.1          5.5         1.8  virginica
## 139          6.0         3.0          4.8         1.8  virginica
## 140          6.9         3.1          5.4         2.1  virginica
## 141          6.7         3.1          5.6         2.4  virginica
## 142          6.9         3.1          5.1         2.3  virginica
## 143          5.8         2.7          5.1         1.9  virginica
## 144          6.8         3.2          5.9         2.3  virginica
## 145          6.7         3.3          5.7         2.5  virginica
## 146          6.7         3.0          5.2         2.3  virginica
## 147          6.3         2.5          5.0         1.9  virginica
## 148          6.5         3.0          5.2         2.0  virginica
## 149          6.2         3.4          5.4         2.3  virginica
## 150          5.9         3.0          5.1         1.8  virginica
\end{verbatim}

\begin{Shaded}
\begin{Highlighting}[]
\NormalTok{sort1 }\OtherTok{\textless{}{-}}\NormalTok{ iris}\SpecialCharTok{$}\NormalTok{Sepal.Length[iris}\SpecialCharTok{$}\NormalTok{Species }\SpecialCharTok{==} \StringTok{\textquotesingle{}setosa\textquotesingle{}}\NormalTok{]}

\NormalTok{sort2 }\OtherTok{\textless{}{-}}\NormalTok{ iris}\SpecialCharTok{$}\NormalTok{Sepal.Length[iris}\SpecialCharTok{$}\NormalTok{Species }\SpecialCharTok{==} \StringTok{\textquotesingle{}versicolor\textquotesingle{}}\NormalTok{]}

\NormalTok{sort3 }\OtherTok{\textless{}{-}}\NormalTok{ iris}\SpecialCharTok{$}\NormalTok{Sepal.Length[iris}\SpecialCharTok{$}\NormalTok{Species }\SpecialCharTok{==} \StringTok{\textquotesingle{}virginica\textquotesingle{}}\NormalTok{]}

\CommentTok{\#checking whether the data is normal distributed}
\FunctionTok{qqnorm}\NormalTok{(sort1)}
\FunctionTok{qqline}\NormalTok{(sort1)}
\end{Highlighting}
\end{Shaded}

\includegraphics{examPrep_files/figure-latex/unnamed-chunk-48-1.pdf}

\begin{Shaded}
\begin{Highlighting}[]
\FunctionTok{shapiro.test}\NormalTok{(sort1)}
\end{Highlighting}
\end{Shaded}

\begin{verbatim}
## 
##  Shapiro-Wilk normality test
## 
## data:  sort1
## W = 0.9777, p-value = 0.4595
\end{verbatim}

\begin{Shaded}
\begin{Highlighting}[]
\FunctionTok{qqnorm}\NormalTok{(sort2)}
\FunctionTok{qqline}\NormalTok{(sort2)}
\end{Highlighting}
\end{Shaded}

\includegraphics{examPrep_files/figure-latex/unnamed-chunk-48-2.pdf}

\begin{Shaded}
\begin{Highlighting}[]
\FunctionTok{shapiro.test}\NormalTok{(sort2)}
\end{Highlighting}
\end{Shaded}

\begin{verbatim}
## 
##  Shapiro-Wilk normality test
## 
## data:  sort2
## W = 0.97784, p-value = 0.4647
\end{verbatim}

\begin{Shaded}
\begin{Highlighting}[]
\FunctionTok{qqnorm}\NormalTok{(sort3)}
\FunctionTok{qqline}\NormalTok{(sort3)}
\end{Highlighting}
\end{Shaded}

\includegraphics{examPrep_files/figure-latex/unnamed-chunk-48-3.pdf}

\begin{Shaded}
\begin{Highlighting}[]
\FunctionTok{shapiro.test}\NormalTok{(sort3)}
\end{Highlighting}
\end{Shaded}

\begin{verbatim}
## 
##  Shapiro-Wilk normality test
## 
## data:  sort3
## W = 0.97118, p-value = 0.2583
\end{verbatim}

\begin{Shaded}
\begin{Highlighting}[]
\CommentTok{\#all three are normally dirstributed}

\CommentTok{\# формула на модела (имаме два отклика)}
\NormalTok{(}\FunctionTok{cbind}\NormalTok{(iris}\SpecialCharTok{$}\NormalTok{Sepal.Length, iris}\SpecialCharTok{$}\NormalTok{Sepal.Width) }\SpecialCharTok{\textasciitilde{}}\NormalTok{ Species) }\SpecialCharTok{\%\textgreater{}\%}
  \CommentTok{\# изпълнение на anova с много у променливи}
  \FunctionTok{manova}\NormalTok{(}\AttributeTok{data =}\NormalTok{ iris) }\SpecialCharTok{\%\textgreater{}\%}
  \CommentTok{\# Обобщение}
  \FunctionTok{summary}\NormalTok{()}
\end{Highlighting}
\end{Shaded}

\begin{verbatim}
##            Df  Pillai approx F num Df den Df    Pr(>F)    
## Species     2 0.94531   65.878      4    294 < 2.2e-16 ***
## Residuals 147                                             
## ---
## Signif. codes:  0 '***' 0.001 '**' 0.01 '*' 0.05 '.' 0.1 ' ' 1
\end{verbatim}

\begin{Shaded}
\begin{Highlighting}[]
  \CommentTok{\# Извод: Различните сортове играят роля за размера на чашелистчетата}
  \CommentTok{\# някоя от групите има значително различно средно от останалите}
\end{Highlighting}
\end{Shaded}

От изпит 2017

\begin{Shaded}
\begin{Highlighting}[]
\CommentTok{\#1.}
\CommentTok{\#number of people younger than 20 yrs}
\FunctionTok{length}\NormalTok{(Aids2}\SpecialCharTok{$}\NormalTok{age[Aids2}\SpecialCharTok{$}\NormalTok{age }\SpecialCharTok{\textless{}} \DecValTok{20}\NormalTok{])}
\end{Highlighting}
\end{Shaded}

\begin{verbatim}
## [1] 39
\end{verbatim}

\begin{Shaded}
\begin{Highlighting}[]
\CommentTok{\#sex of the patients with earliest diagnosis}
\NormalTok{Aids2}\SpecialCharTok{$}\NormalTok{sex[}\FunctionTok{head}\NormalTok{(}\FunctionTok{order}\NormalTok{(Aids2}\SpecialCharTok{$}\NormalTok{diag), }\DecValTok{5}\NormalTok{)]}
\end{Highlighting}
\end{Shaded}

\begin{verbatim}
## [1] M M M M M
## Levels: F M
\end{verbatim}

\begin{Shaded}
\begin{Highlighting}[]
\CommentTok{\#men who got aids from blood}
\NormalTok{men\_blood }\OtherTok{\textless{}{-}} \FunctionTok{sum}\NormalTok{(Aids2}\SpecialCharTok{$}\NormalTok{sex[Aids2}\SpecialCharTok{$}\NormalTok{T.categ }\SpecialCharTok{==} \StringTok{\textquotesingle{}blood\textquotesingle{}}\NormalTok{] }\SpecialCharTok{==} \StringTok{\textquotesingle{}M\textquotesingle{}}\NormalTok{)}

\NormalTok{all\_men }\OtherTok{\textless{}{-}} \FunctionTok{sum}\NormalTok{(Aids2}\SpecialCharTok{$}\NormalTok{sex }\SpecialCharTok{==} \StringTok{\textquotesingle{}M\textquotesingle{}}\NormalTok{)}

\NormalTok{men\_blood }\SpecialCharTok{/}\NormalTok{ all\_men}
\end{Highlighting}
\end{Shaded}

\begin{verbatim}
## [1] 0.02069717
\end{verbatim}

\begin{Shaded}
\begin{Highlighting}[]
\CommentTok{\#графика за щатът на пациента и смъртността}
\FunctionTok{plot}\NormalTok{(Aids2}\SpecialCharTok{$}\NormalTok{state, Aids2}\SpecialCharTok{$}\NormalTok{death, }\AttributeTok{na.rm =}\NormalTok{ T)}
\end{Highlighting}
\end{Shaded}

\includegraphics{examPrep_files/figure-latex/unnamed-chunk-49-1.pdf}

\begin{Shaded}
\begin{Highlighting}[]
\FunctionTok{table}\NormalTok{(Aids2}\SpecialCharTok{$}\NormalTok{status, Aids2}\SpecialCharTok{$}\NormalTok{state) }\SpecialCharTok{\%\textgreater{}\%} \FunctionTok{prop.table}\NormalTok{() }\SpecialCharTok{\%\textgreater{}\%} \FunctionTok{barplot}\NormalTok{(}\AttributeTok{legend.text =}\NormalTok{ T)}
\end{Highlighting}
\end{Shaded}

\includegraphics{examPrep_files/figure-latex/unnamed-chunk-49-2.pdf}

\begin{Shaded}
\begin{Highlighting}[]
\NormalTok{?Aids2}
\end{Highlighting}
\end{Shaded}

\begin{Shaded}
\begin{Highlighting}[]
\CommentTok{\#2.}
\CommentTok{\#total women and women that died}
\NormalTok{women }\OtherTok{\textless{}{-}} \FunctionTok{sum}\NormalTok{(Aids2}\SpecialCharTok{$}\NormalTok{sex }\SpecialCharTok{==} \StringTok{\textquotesingle{}F\textquotesingle{}}\NormalTok{)}
\NormalTok{dead\_women }\OtherTok{\textless{}{-}} \FunctionTok{sum}\NormalTok{(Aids2}\SpecialCharTok{$}\NormalTok{status[Aids2}\SpecialCharTok{$}\NormalTok{sex }\SpecialCharTok{==} \StringTok{\textquotesingle{}F\textquotesingle{}}\NormalTok{] }\SpecialCharTok{==} \StringTok{\textquotesingle{}D\textquotesingle{}}\NormalTok{)}

\NormalTok{men }\OtherTok{\textless{}{-}} \FunctionTok{sum}\NormalTok{(Aids2}\SpecialCharTok{$}\NormalTok{sex }\SpecialCharTok{==} \StringTok{\textquotesingle{}M\textquotesingle{}}\NormalTok{)}
\NormalTok{dead\_men }\OtherTok{\textless{}{-}} \FunctionTok{sum}\NormalTok{(Aids2}\SpecialCharTok{$}\NormalTok{status[Aids2}\SpecialCharTok{$}\NormalTok{sex }\SpecialCharTok{==} \StringTok{\textquotesingle{}M\textquotesingle{}}\NormalTok{] }\SpecialCharTok{==} \StringTok{\textquotesingle{}D\textquotesingle{}}\NormalTok{)}

\CommentTok{\#Х0 {-} жените умират по{-}малко Х1 {-} умират повече}
\FunctionTok{prop.test}\NormalTok{(}\FunctionTok{c}\NormalTok{(dead\_women, dead\_men), }\FunctionTok{c}\NormalTok{(women, men), }\AttributeTok{alternative =} \StringTok{\textquotesingle{}greater\textquotesingle{}}\NormalTok{)}
\end{Highlighting}
\end{Shaded}

\begin{verbatim}
## 
##  2-sample test for equality of proportions with continuity correction
## 
## data:  c(dead_women, dead_men) out of c(women, men)
## X-squared = 0.13041, df = 1, p-value = 0.641
## alternative hypothesis: greater
## 95 percent confidence interval:
##  -0.1173963  1.0000000
## sample estimates:
##    prop 1    prop 2 
## 0.5955056 0.6201888
\end{verbatim}

\begin{Shaded}
\begin{Highlighting}[]
\CommentTok{\#можем да приемем нулевата хипотеза}
\end{Highlighting}
\end{Shaded}

\begin{Shaded}
\begin{Highlighting}[]
\NormalTok{number\_dead }\OtherTok{\textless{}{-}}\NormalTok{ Aids2}\SpecialCharTok{$}\NormalTok{age[Aids2}\SpecialCharTok{$}\NormalTok{status }\SpecialCharTok{==} \StringTok{\textquotesingle{}D\textquotesingle{}}\NormalTok{]}

\FunctionTok{qqnorm}\NormalTok{(number\_dead)}
\FunctionTok{qqline}\NormalTok{(number\_dead)}
\end{Highlighting}
\end{Shaded}

\includegraphics{examPrep_files/figure-latex/unnamed-chunk-51-1.pdf}

\begin{Shaded}
\begin{Highlighting}[]
\FunctionTok{shapiro.test}\NormalTok{(number\_dead)}
\end{Highlighting}
\end{Shaded}

\begin{verbatim}
## 
##  Shapiro-Wilk normality test
## 
## data:  number_dead
## W = 0.96911, p-value < 2.2e-16
\end{verbatim}

\begin{Shaded}
\begin{Highlighting}[]
\CommentTok{\#not normally distributed}

\CommentTok{\#Х0 {-} средната възраст е 38 Х1 {-} не е }
\CommentTok{\#приемаме h1 хипотеза}
\FunctionTok{wilcox.test}\NormalTok{(number\_dead, }\AttributeTok{mu =} \DecValTok{38}\NormalTok{, }\AttributeTok{alternative =} \StringTok{\textquotesingle{}two.sided\textquotesingle{}}\NormalTok{)}
\end{Highlighting}
\end{Shaded}

\begin{verbatim}
## 
##  Wilcoxon signed rank test with continuity correction
## 
## data:  number_dead
## V = 649726, p-value = 0.00164
## alternative hypothesis: true location is not equal to 38
\end{verbatim}

\begin{Shaded}
\begin{Highlighting}[]
\CommentTok{\#4.}
\NormalTok{X }\OtherTok{\textless{}{-}} \FunctionTok{rchisq}\NormalTok{(}\DecValTok{100}\NormalTok{, }\AttributeTok{df =} \DecValTok{10}\NormalTok{)}

\FunctionTok{hist}\NormalTok{(X, }\AttributeTok{probability =}\NormalTok{ T)}

\FunctionTok{lines}\NormalTok{(}\FunctionTok{dchisq}\NormalTok{(}\DecValTok{0}\SpecialCharTok{:}\DecValTok{30}\NormalTok{, }\AttributeTok{df =} \DecValTok{10}\NormalTok{))}
\end{Highlighting}
\end{Shaded}

\includegraphics{examPrep_files/figure-latex/unnamed-chunk-52-1.pdf}

\begin{Shaded}
\begin{Highlighting}[]
\CommentTok{\#5.}
\NormalTok{cat }\OtherTok{\textless{}{-}}\NormalTok{ cats[cats}\SpecialCharTok{$}\NormalTok{Sex }\SpecialCharTok{==} \StringTok{\textquotesingle{}M\textquotesingle{}}\NormalTok{, ]}

\CommentTok{\#create the model}
\NormalTok{s }\OtherTok{\textless{}{-}} \FunctionTok{lm}\NormalTok{(Hwt }\SpecialCharTok{\textasciitilde{}}\NormalTok{ Bwt, }\AttributeTok{data =}\NormalTok{ cat) }\SpecialCharTok{\%\textgreater{}\%} \FunctionTok{summary}\NormalTok{()}
\CommentTok{\#based on the model we get that the heart and body weight are not independant }
\CommentTok{\#демек са зависими и сърцето се повлиява от теглото на котката}

\NormalTok{t }\OtherTok{\textless{}{-}}\NormalTok{ (s}\SpecialCharTok{$}\NormalTok{coefficients[}\DecValTok{2}\NormalTok{, }\DecValTok{1}\NormalTok{] }\SpecialCharTok{{-}} \DecValTok{5}\NormalTok{) }\SpecialCharTok{/}\NormalTok{ s}\SpecialCharTok{$}\NormalTok{coefficients[}\DecValTok{2}\NormalTok{, }\DecValTok{2}\NormalTok{]}
\CommentTok{\#t is negative =\textgreater{} we calculate p{-}value like this:}
\CommentTok{\#df is equal to: numOfObservations {-} numOfEvaluatedParameters {-} 1}
\CommentTok{\#            so: 97 {-} 2 {-} 1 = 94}
\CommentTok{\#Вярно ли е, че при котки по тежки с 1 кг сърцето е по тежко с 5 гр {-} H0}
\NormalTok{pval }\OtherTok{=} \DecValTok{2} \SpecialCharTok{*} \FunctionTok{pt}\NormalTok{(t, }\AttributeTok{df =} \DecValTok{94}\NormalTok{)}
\CommentTok{\#тук имаме p{-}value \textless{} 0.05 което значи че отхвърляме нулевата хипотеза, т.е горното не е вярно}

\CommentTok{\# pval = 2 * pt(t, df = 94, lower.tail = F) if t \textgreater{} 0 only lower tail}

\CommentTok{\#we test if the distribution is normal so we can do t.test to get the conf.interval}
\FunctionTok{shapiro.test}\NormalTok{(cat}\SpecialCharTok{$}\NormalTok{Hwt[cat}\SpecialCharTok{$}\NormalTok{Bwt }\SpecialCharTok{==} \FloatTok{2.6}\NormalTok{])}
\end{Highlighting}
\end{Shaded}

\begin{verbatim}
## 
##  Shapiro-Wilk normality test
## 
## data:  cat$Hwt[cat$Bwt == 2.6]
## W = 0.96653, p-value = 0.8683
\end{verbatim}

\begin{Shaded}
\begin{Highlighting}[]
\FunctionTok{hist}\NormalTok{(cat}\SpecialCharTok{$}\NormalTok{Hwt[cat}\SpecialCharTok{$}\NormalTok{Bwt }\SpecialCharTok{==} \FloatTok{2.6}\NormalTok{])}
\end{Highlighting}
\end{Shaded}

\includegraphics{examPrep_files/figure-latex/unnamed-chunk-53-1.pdf}

\begin{Shaded}
\begin{Highlighting}[]
\CommentTok{\#it is norm dist so we use t.test else we use wilcox.test()}
\FunctionTok{t.test}\NormalTok{(cat}\SpecialCharTok{$}\NormalTok{Hwt[cat}\SpecialCharTok{$}\NormalTok{Bwt }\SpecialCharTok{==} \FloatTok{2.6}\NormalTok{], }\AttributeTok{conf.level =} \FloatTok{0.95}\NormalTok{)}
\end{Highlighting}
\end{Shaded}

\begin{verbatim}
## 
##  One Sample t-test
## 
## data:  cat$Hwt[cat$Bwt == 2.6]
## t = 16.654, df = 5, p-value = 1.426e-05
## alternative hypothesis: true mean is not equal to 0
## 95 percent confidence interval:
##   8.005474 10.927859
## sample estimates:
## mean of x 
##  9.466667
\end{verbatim}

От примерен тест 2017

\begin{Shaded}
\begin{Highlighting}[]
\CommentTok{\#1.}
\FunctionTok{qnorm}\NormalTok{(}\AttributeTok{p =} \FloatTok{0.05}\NormalTok{)}
\end{Highlighting}
\end{Shaded}

\begin{verbatim}
## [1] -1.644854
\end{verbatim}

\begin{Shaded}
\begin{Highlighting}[]
\CommentTok{\#2.}
\FunctionTok{nrow}\NormalTok{(state.x77)}
\end{Highlighting}
\end{Shaded}

\begin{verbatim}
## [1] 50
\end{verbatim}

\begin{Shaded}
\begin{Highlighting}[]
\CommentTok{\#подреждаме щатите по ниво на необразованост}
\NormalTok{dumb\_states }\OtherTok{\textless{}{-}} \FunctionTok{head}\NormalTok{(}\FunctionTok{order}\NormalTok{(state.x77[,}\DecValTok{3}\NormalTok{], }\AttributeTok{decreasing =}\NormalTok{ T), }\DecValTok{5}\NormalTok{)}
\CommentTok{\#взимаме ги според индексите на първите 5 щата}
\NormalTok{state.x77[dumb\_states, }\DecValTok{3}\NormalTok{]}
\end{Highlighting}
\end{Shaded}

\begin{verbatim}
##      Louisiana    Mississippi South Carolina     New Mexico          Texas 
##            2.8            2.4            2.3            2.2            2.2
\end{verbatim}

\begin{Shaded}
\begin{Highlighting}[]
\CommentTok{\#states with life expectancy over 70}
\NormalTok{old\_states }\OtherTok{\textless{}{-}}\NormalTok{ state.x77[}\DecValTok{1}\SpecialCharTok{:}\DecValTok{50}\NormalTok{, }\DecValTok{4}\NormalTok{] }\SpecialCharTok{\textgreater{}} \DecValTok{70}
\FunctionTok{length}\NormalTok{(state.x77[old\_states, }\DecValTok{4}\NormalTok{])}
\end{Highlighting}
\end{Shaded}

\begin{verbatim}
## [1] 41
\end{verbatim}

\begin{Shaded}
\begin{Highlighting}[]
\CommentTok{\#щат с най{-}голяма гъстота на населението}
\NormalTok{pop }\OtherTok{\textless{}{-}}\NormalTok{ state.x77[}\DecValTok{1}\SpecialCharTok{:}\DecValTok{50}\NormalTok{,}\DecValTok{1}\NormalTok{]}

\NormalTok{land }\OtherTok{\textless{}{-}}\NormalTok{ state.x77[}\DecValTok{1}\SpecialCharTok{:}\DecValTok{50}\NormalTok{, }\DecValTok{8}\NormalTok{]}

\NormalTok{density }\OtherTok{\textless{}{-}}\NormalTok{ pop }\SpecialCharTok{/}\NormalTok{ land}

\NormalTok{density[}\FunctionTok{head}\NormalTok{(}\FunctionTok{order}\NormalTok{(density, }\AttributeTok{decreasing =}\NormalTok{ T), }\DecValTok{1}\NormalTok{)]}
\end{Highlighting}
\end{Shaded}

\begin{verbatim}
## New Jersey 
##  0.9750033
\end{verbatim}

\begin{Shaded}
\begin{Highlighting}[]
\CommentTok{\#общото население на петте най{-}големи щати}
\NormalTok{biggest\_states }\OtherTok{\textless{}{-}} \FunctionTok{head}\NormalTok{(}\FunctionTok{order}\NormalTok{(state.x77[}\DecValTok{1}\SpecialCharTok{:}\DecValTok{50}\NormalTok{, }\DecValTok{8}\NormalTok{], }\AttributeTok{decreasing =}\NormalTok{ T), }\DecValTok{5}\NormalTok{)}
\FunctionTok{sum}\NormalTok{(state.x77[biggest\_states, }\DecValTok{1}\NormalTok{])}
\end{Highlighting}
\end{Shaded}

\begin{verbatim}
## [1] 35690
\end{verbatim}

\begin{Shaded}
\begin{Highlighting}[]
\CommentTok{\#3.ho {-} има по{-}малко подобрили се жени х1{-} има повече подобрили се мъже}
\NormalTok{women }\OtherTok{\textless{}{-}} \DecValTok{200}
\NormalTok{men }\OtherTok{\textless{}{-}} \DecValTok{100}

\NormalTok{not\_accepted\_women }\OtherTok{\textless{}{-}}\NormalTok{ women }\SpecialCharTok{*} \DecValTok{38} \SpecialCharTok{/} \DecValTok{100}
\NormalTok{not\_accepted\_men }\OtherTok{\textless{}{-}}\NormalTok{ men }\SpecialCharTok{*} \DecValTok{50} \SpecialCharTok{/} \DecValTok{100}

\CommentTok{\#приемаме хипотезата че е по{-}ефективно при жените отколкото при мъжете}
\FunctionTok{prop.test}\NormalTok{(}\FunctionTok{c}\NormalTok{(not\_accepted\_women, not\_accepted\_men), }\FunctionTok{c}\NormalTok{(women, men), }\AttributeTok{alternative =} \StringTok{\textquotesingle{}greater\textquotesingle{}}\NormalTok{ )}
\end{Highlighting}
\end{Shaded}

\begin{verbatim}
## 
##  2-sample test for equality of proportions with continuity correction
## 
## data:  c(not_accepted_women, not_accepted_men) out of c(women, men)
## X-squared = 3.4637, df = 1, p-value = 0.9686
## alternative hypothesis: greater
## 95 percent confidence interval:
##  -0.2272546  1.0000000
## sample estimates:
## prop 1 prop 2 
##   0.38   0.50
\end{verbatim}

\begin{Shaded}
\begin{Highlighting}[]
\CommentTok{\#4.}
\NormalTok{ data }\OtherTok{\textless{}{-}} \FunctionTok{data.frame}\NormalTok{(}
\NormalTok{anscombe}\SpecialCharTok{$}\NormalTok{x3,}
\NormalTok{anscombe}\SpecialCharTok{$}\NormalTok{x4)}

\NormalTok{l }\OtherTok{\textless{}{-}} \FunctionTok{lm}\NormalTok{(data}\SpecialCharTok{$}\NormalTok{anscombe.x3 }\SpecialCharTok{\textasciitilde{}}\NormalTok{ data}\SpecialCharTok{$}\NormalTok{anscombe.x4, }\AttributeTok{data =}\NormalTok{ data) }

\FunctionTok{plot}\NormalTok{(data}\SpecialCharTok{$}\NormalTok{anscombe.x3, data}\SpecialCharTok{$}\NormalTok{anscombe.x4)}
\FunctionTok{abline}\NormalTok{(l, }\AttributeTok{col =} \StringTok{"red "}\NormalTok{, }\AttributeTok{lwd =} \DecValTok{2}\NormalTok{)}
\end{Highlighting}
\end{Shaded}

\includegraphics{examPrep_files/figure-latex/unnamed-chunk-57-1.pdf}

\end{document}
